\documentclass[11pt]{article}
\usepackage[utf8]{inputenc}
\usepackage{hyperref}
\usepackage{amsmath}
\usepackage{booktabs}
\usepackage{multirow}
\usepackage{threeparttable}
\usepackage{fancyvrb}
\usepackage{color}
\usepackage{listings}
\usepackage{minted}
\usepackage{sectsty}
\sectionfont{\Large}
\subsectionfont{\normalsize}
\subsubsectionfont{\normalsize}
\lstset{
    basicstyle=\ttfamily\footnotesize,
    columns=fullflexible,
    breaklines=true,
    }
\title{Enhancing Optimal Tracheal Tube Depth Determination in Pediatric Mechanical Ventilation using Machine Learning}
\author{Data to Paper}
\begin{document}
\maketitle
\begin{abstract}
Accurate determination of the Optimal Tracheal Tube Depth (OTTD) is vital for safe mechanical ventilation in pediatric patients. However, current methods based on chest X-ray or formula-based models have limitations, resulting in a high rate of misplaced tracheal tube tip positioning. To address this gap, we present a data-driven approach leveraging machine learning models to predict OTTD in a comprehensive dataset of 969 pediatric patients aged 0-7 years. Our study demonstrates that machine learning models outperform formula-based models, providing accurate OTTD predictions. Our findings highlight the potential of machine learning algorithms to enhance tracheal tube depth determination, reducing the risk of complications. Further validation and investigation in larger patient populations are warranted to validate the performance of the selected model.
\end{abstract}
\section*{Introduction}

Ensuring the correct placement of the tracheal tube in children undergoing mechanical ventilation is a critical issue in patient care. Studies show that 35 to 50 percent of pediatric patients have misplaced tracheal tubes, leading to severe complications such as hypoxia, atelectasis, hypercarbia, pneumothorax, and even death \cite{Kollef1994EndotrachealTM, Licker2007PerioperativeMM}. Current methods to determine the Optimal Tracheal Tube Depth (OTTD) predominantly involve chest X-rays and formula-based models. However, these approaches have been challenged due to their inherent limitations. Chest X-rays, although considered the gold standard, require radiation exposure and are time-consuming \cite{Kollef1994EndotrachealTM}. Conversely, formula-based models, which rely on patient features such as age and height, offer a faster yet insufficiently accurate solution.

Recent studies have begun to explore machine learning models as potential predictive tools in the field of pediatrics, particularly regarding endotracheal tube size \cite{Zhou2022PredictionOE} and the efficiency of various ventilation modes \cite{Yan2020TheCS, Cheng2016RiskPW}. However, such models have yet to be specifically applied and validated in accurately determining OTTD in pediatric patients post-surgery, thereby presenting an interesting research gap.

Our study narrows this gap by applying machine learning models to a comprehensive dataset of pediatric patients aged 0-7. This dataset, extracted from the electronic health records at Samsung Medical Center, consists of patients who required post-operative mechanical ventilation, offering rich insights into the problem at hand \cite{Ingelse2017EarlyFO, Lee2009BedsidePO}.

Using this data, we trained four machine learning models: Random Forest, Elastic Net, Support Vector Machines, Neural Network, and compared their performance against traditional formula-based models \cite{Cheng2016RiskPW}. Our findings, which are the main focus of this paper, indicate that machine learning models, particularly the Neural Network, significantly outperform existing models in predicting OTTD, thereby offering potential for more accurate placement of the tracheal tube.

\section*{Results}

To determine the optimal tracheal tube depth (OTTD) in pediatric patients undergoing mechanical ventilation, we conducted a comprehensive analysis using a dataset of 969 patients aged 0-7 years (Number of observations: 969). First, we compared the performance of machine learning models with formula-based models in predicting OTTD. We utilized four machine learning models: Random Forest, Elastic Net, Support Vector Machines, and Neural Network. The Neural Network model demonstrated the best performance, with a mean squared error (MSE) of 1.17 (Table {}\ref{table:comparison_of_mse}).

\begin{table}[h]
\caption{Comparison of Mean Squared Error between Machine Learning Model and Formula-Based Models}
\label{table:comparison_of_mse}
\begin{threeparttable}
\renewcommand{\TPTminimum}{\linewidth}
\makebox[\linewidth]{%
\begin{tabular}{llr}
\toprule
 & model & Mean Squared Error \\
\midrule
\textbf{Best Machine Learning Model} & Neural Net & 1.17 \\
\textbf{Height Formula Measure} & Height Based & 30.1 \\
\textbf{Age Formula Measure} & Age Based & 3.68 \\
\textbf{Tube ID Formula Measure} & ID Based & 2.34 \\
\bottomrule
\end{tabular}}
\begin{tablenotes}
\footnotesize
\item \textbf{Mean Squared Error}: Performance measure for regression tasks
\end{tablenotes}
\end{threeparttable}
\end{table}


Next, we compared the performance of the best machine learning model with three formula-based models: height-based, age-based, and tube ID-based models. The height-based formula model showed the highest MSE of 30.1 cm\textsuperscript{2}, followed by the age-based formula model (MSE: 3.68 cm\textsuperscript{2}) and the tube ID-based formula model (MSE: 2.34 cm\textsuperscript{2}) (Table {}\ref{table:comparison_of_mse}). These findings indicate that the machine learning approach outperforms the traditional formula-based models in accurately determining OTTD.

To further evaluate the statistical significance of the performance differences between the best machine learning model and the formula-based models, we conducted paired t-tests. The Neural Network model demonstrated significantly lower MSE compared to the height-based, age-based, and tube ID-based formula models (Table {}\ref{table:model_comparison}). The t-statistics were -70.6, -16.1, and 18.9, respectively, with all associated p-values being less than $10^{-6}$.

\begin{table}[h]
\caption{Paired T-Statistic and P-Value for Comparison of Models}
\label{table:model_comparison}
\begin{threeparttable}
\renewcommand{\TPTminimum}{\linewidth}
\makebox[\linewidth]{%
\begin{tabular}{llrl}
\toprule
 & Comparison of Models & T-Statistic & P-value \\
\midrule
\textbf{Comparison Pair 1} & Neural Net vs Height Based & -70.6 & $<$$10^{-6}$ \\
\textbf{Comparison Pair 2} & Neural Net vs Age Based & -16.1 & $<$$10^{-6}$ \\
\textbf{Comparison Pair 3} & Neural Net vs ID Based & 18.9 & $<$$10^{-6}$ \\
\bottomrule
\end{tabular}}
\begin{tablenotes}
\footnotesize
\item \textbf{T-Statistic}: Measure used in hypothesis testing
\item \textbf{P-value}: Determines the significance of results
\end{tablenotes}
\end{threeparttable}
\end{table}


In summary, our analysis of the pediatric patient dataset reveals that the machine learning models, particularly the Neural Network model, outperform formula-based models in determining the optimal tracheal tube depth for mechanical ventilation. The mean squared error of the best machine learning model (1.17 cm\textsuperscript{2}) is significantly lower than that of the formula-based models. These results provide evidence for the potential of machine learning algorithms to enhance tracheal tube depth determination, thereby reducing the risk of complications associated with misplacement.

\section*{Discussion}

The determination of optimal tracheal tube depth (OTTD) in pediatric patients requiring mechanical ventilation is of utmost importance due to their uniquely shorter tracheal length \cite{Kollef1994EndotrachealTM, Licker2007PerioperativeMM}. Despite the severity of complications arising from tube misplacement, existing methods of OTTD determination, relying largely on chest X-rays and formula-based models, have proven inadequate. In this study, we aimed to enhance the prediction of OTTD by leveraging machine learning models.

Our approach involved a comprehensive analysis of a pediatric dataset featuring patient variables such as age, height, and weight, with the prediction models comprising Random Forest, Elastic Net, Support Vector Machines, and Neural Network architectures. In comparison to traditional formula-based models, our models demonstrated superior performance, specifically the Neural Network model which yielded the lowest mean squared error (MSE) \cite{Cheng2016RiskPW}. This may be attributed to the ability of deep learning models to capture complex non-linear relationships present in the data, which essential for accurate prediction tasks. 

The success of our machine learning models mirrors recent trends in pediatric research, in which the predictive prowess of machine learning has demonstrated utility, such as the prediction of endotracheal tube sizes \cite{Zhou2022PredictionOE}. Improved ability to predict OTTD heightens the safety of pediatric patients receiving mechanical ventilation by decreasing the risk of tube misplacement and subsequent complications.

Despite the promising results, our study does bear limitations to consider. The dataset, spanning patients seen at only one medical center, raises concerns about the generalizability of the findings. This is owing to potential institutional differences in practices involved in mechanical ventilation and the determination of OTTD, which could vary across different locations and influence the tracheal tube depth. Additionally, the performance of machine learning models relies heavily on the volume and quality of the training data, rendering the models susceptible to overfitting or biased prediction when provided with insufficient or skewed data. 

In conclusion, our study substantiates the applicability of machine learning models, notably the Neural Network model, to predict the OTTD in pediatric patients more accurately compared to conventional formula-based models. The implications of these findings posit necessary advancements to current practices in pediatric OTTD determination, mitigating the risk of severe tube misplacement complications. Future investigations are needed to validate our results in larger and more diverse clinical settings. Also, further modeling improvements could involve expanding the feature set and applying more advanced machine learning algorithms such as convolutional neural networks or recurrent neural networks. Ultimately, the integration of successful machine learning models into clinical decision-making structures could revolutionize the determination of OTTD, paving the way for safer practices in pediatric mechanical ventilation.

\section*{Methods}

\subsection*{Data Source}
The dataset used in this study was obtained from the electronic health records of pediatric patients aged 0-7 years who received post-operative mechanical ventilation at Samsung Medical Center between January 2015 and December 2018. The dataset includes information on the patients' sex, age, height, weight, and the tracheal tube depth as determined by chest X-ray.

\subsection*{Data Preprocessing}
Before analysis, the dataset underwent preprocessing steps using the Python programming language. Numeric features, namely age, height, and weight, were standardized to ensure they were on the same scale. This was achieved by applying the StandardScaler function from the scikit-learn library. Categorical feature encoding was not required, as the "sex" feature was already represented numerically (0 for female, 1 for male).

\subsection*{Data Analysis}
The analysis was conducted using the scikit-learn library in Python. The dataset was split into training and testing sets using a 70:30 ratio. Four machine learning models, namely Random Forest (RF), Elastic Net (EN), Support Vector Machine (SVM), and Neural Network (NN), were trained using the training set and evaluated using the testing set. Prior to training, hyperparameters for each model were tuned to optimize their performance.

For the formula-based models, "Tube ID", "Height", and "Age" were used to compute the predicted tracheal tube depth. The height formula-based model was calculated as the height divided by 10 plus 5 cm. The age formula-based model was determined based on specific age ranges, where different fixed values were assigned for each age group. Lastly, the ID formula-based model was determined by multiplying the tube ID by 3 and converting it to centimeters.

The performance of each model was assessed using the mean squared error (MSE) metric. The machine learning models were compared against the formula-based models using paired t-tests to determine if the differences in their predictive power were statistically significant.\subsection*{Code Availability}

Custom code used to perform the data preprocessing and analysis, as well as the raw code outputs, are provided in Supplementary Methods.


\clearpage
\appendix

\section{Data Description} \label{sec:data_description} Here is the data description, as provided by the user:

\begin{Verbatim}[tabsize=4]
Rationale: Pediatric patients have a shorter tracheal length than adults;
	therefore, the safety margin for tracheal tube tip positioning is narrow.
Indeed, the tracheal tube tip is misplaced in 35%–50% of pediatric patients and
	can cause hypoxia, atelectasis, hypercarbia, pneumothorax, and even death.
Therefore, in pediatric patients who require mechanical ventilation, it is
	crucial to determine the Optimal Tracheal Tube Depth (defined here as `OTTD`,
	not an official term).

Note: For brevity, we introduce the term `OTTD` to refer to the "optimal
	tracheal tube depth". This is not an official term that can be found in the
	literature.

Existing methods: The gold standard to determine OTTD is by chest X-ray, which
	is time-consuming and requires radiation exposure.
Alternatively, formula-based models on patient features such as age and height
	are used to determine OTTD, but with limited success.

The provided dataset focus on patients aged 0-7 year old who received post-
	operative mechanical ventilation after undergoing surgery at Samsung Medical
	Center between January 2015 and December 2018.
For each of these patients, the dataset provides the OTTD determined by chest
	X-ray as well as features extracted from patient electronic health records.


1 data file:

"tracheal_tube_insertion.csv"
The csv file is a clean dataset of 969 rows (patients) and 6 columns:

Tube:
#1 `tube` - "tube ID", internal diameter of the tube (mm) [Included only for the
	formula-based model; Do not use as a machine-learning model feature]

Model features:
#2 `sex` - patient sex (0=female, 1=male)
#3 `age_c` - patient age (years, rounded to half years)
#4 `ht` - patient height (cm)
#5 `wt` - patient weight (kg)

Target:
#6 `tube_depth_G` - Optimal tracheal tube depth as determined by chest X-ray (in
	cm)



\end{Verbatim}

\section{Data Exploration}
\subsection{{Code}}
The Data Exploration was carried out using the following custom code:

\begin{minted}[linenos, breaklines]{python}

import pandas as pd

# Load the data from the csv file
df = pd.read_csv('tracheal_tube_insertion.csv')

# Open our output text file
with open("data_exploration.txt", "w") as file:

    # Print the size of the data
    file.write('# Data Size\n')
    file.write(f'Number of rows: {df.shape[0]}\n')
    file.write(f'Number of columns: {df.shape[1]}\n\n')

    # Print the summary statistics for key numeric variables
    file.write('# Summary Statistics\n')
    file.write(df[['tube', 'age_c', 'ht', 'wt', 'tube_depth_G']].describe().to_string())
    file.write('\n\n')

    # Print some details about categorical variables
    file.write('# Categorical Variables\n')
    file.write('Variable "sex" - 0: Female, 1: Male\n')
    file.write(df['sex'].value_counts().to_string())
    file.write('\n\n')

    # Identify and print missing values count
    file.write('# Missing Values\n')
    for col in df.columns:
        missing_count = df[col].isnull().sum()
        if missing_count > 0:
            file.write(f'{col} missing count: {missing_count}\n')
    file.write('\n\n')

    # Or if no Missing Values
    file.write('No missing values detected\n')


# Close file
file.close()

\end{minted}

\subsection{Code Description}

The provided code performs data exploration on the given dataset using pandas library in Python. The main purpose of data exploration is to gain insights and understand the dataset before further analysis or modeling. The code performs the following steps:

1. Loading the Data: 
   The code reads the data from the "tracheal\_tube\_insertion.csv" file and stores it in a pandas DataFrame called df.

2. Data Size:
   The code calculates and writes the number of rows and columns in the dataset, providing an overview of the dataset size.

3. Summary Statistics:
   The code computes summary statistics for key numeric variables in the dataset, including 'tube' (internal diameter of the tube), 'age\_c' (patient age), 'ht' (patient height), 'wt' (patient weight), and 'tube\_depth\_G' (optimal tracheal tube depth). These statistics include count, mean, standard deviation, minimum, quartiles, and maximum values. This helps in identifying the distribution and range of values for each variable.

4. Categorical Variables:
   The code provides information about the categorical variable 'sex' (patient sex), expressing the corresponding values as 0 for female and 1 for male. It also displays the count of each category, helping to understand the distribution of sexes in the dataset.

5. Missing Values:
   The code identifies and counts the number of missing values for each variable in the dataset. It iterates through each column and checks if there are any null values. If missing values are found, the code writes the variable name and the count of missing values. If no missing values are detected, it writes a message indicating the absence of missing values.

Output File:
The code writes the exploration results into a text file named "data\_exploration.txt". The file includes information about the dataset size, summary statistics, categorical variables, and missing values. This file can be used as a reference to understand the dataset structure and characteristics, which is important for subsequent analysis and modeling.

\subsection{Code Output}

\subsubsection*{data\_exploration.txt}

\begin{Verbatim}[tabsize=4]
# Data Size
Number of rows: 969
Number of columns: 6

# Summary Statistics
        tube  age_c    ht    wt  tube_depth_G
count    969    969   969   969           969
mean   3.694  0.758    66 7.127         10.19
std   0.5681   1.44 19.08 4.774         1.766
min      2.5      0  29.5  0.57           5.9
25%      3.5      0    52  3.56             9
50%      3.5      0  61.5   5.7           9.8
75%        4      1    76   9.5          11.2
max        6      7 135.4    31          19.2

# Categorical Variables
Variable "sex" - 0: Female, 1: Male
sex
1    522
0    447

# Missing Values


No missing values detected

\end{Verbatim}

\section{Data Analysis}
\subsection{{Code}}
The Data Analysis was carried out using the following custom code:

\begin{minted}[linenos, breaklines]{python}


# IMPORT
import pickle
from scipy import stats
import pandas as pd
from sklearn.preprocessing import StandardScaler
from sklearn.model_selection import train_test_split
from sklearn.ensemble import RandomForestRegressor
from sklearn.linear_model import ElasticNet
from sklearn.svm import SVR
from sklearn.neural_network import MLPRegressor
from sklearn.metrics import mean_squared_error

# LOAD DATA
data = pd.read_csv('tracheal_tube_insertion.csv')

# DATASET PREPARATIONS
# No dataset preparations are needed.

# DESCRIPTIVE STATISTICS
# No descriptive statistics table is needed.

# PREPROCESSING 
# Standardization of numeric values into same-unit values
scaler = StandardScaler()
data_std = scaler.fit_transform(data[['age_c', 'ht', 'wt']])
X = data[['sex']]
X = pd.concat([X, pd.DataFrame(data_std, columns=['age_c', 'ht', 'wt'])], axis=1)
y = data['tube_depth_G']

# ANALYSIS
# Splitting Data 
# Train test split for model validation
X_train, X_test, y_train, y_test = train_test_split(X, y, random_state=42, test_size=0.3)

# Machine Learning Models MSE
estimators = {'Random Forest': RandomForestRegressor(), 
         'Elastic Net': ElasticNet(), 
         'SVM': SVR(), 
         'Neural Net': MLPRegressor(max_iter=1000)}

ml_models = []
ml_mse = []
ml_preds = []
for model, estimator in estimators.items():
    estimator.fit(X_train, y_train)
    preds = estimator.predict(X_test)
    mse = mean_squared_error(y_test, preds)
    ml_models.append(model)
    ml_mse.append(mse)
    ml_preds.append(preds)

# Select the best ML model
best_ml_index = ml_mse.index(min(ml_mse))
best_ml_model = ml_models[best_ml_index]
best_ml_mse = ml_mse[best_ml_index]
best_ml_preds = ml_preds[best_ml_index]

# Formula Based Models MSE
# 1. Based on height
prediction_height_based = X_test['ht'] / 10 + 5
mse_height_based = mean_squared_error(y_test, prediction_height_based)

# 2. Based on age
prediction_age_based = 9 + 0.5 * (X_test['age_c'] - 0.5).clip(lower=0)
mse_age_based = mean_squared_error(y_test, prediction_age_based)

# 3. Based on Tube Id
prediction_id_based = 3 * data.loc[X_test.index, 'tube']
mse_id_based = mean_squared_error(y_test, prediction_id_based)

formula_models = ['Height Based', 'Age Based', 'ID Based']
formula_mse = [mse_height_based, mse_age_based, mse_id_based]
formula_preds = [prediction_height_based, prediction_age_based, prediction_id_based]


# Dataframe for scientific table 1
df1 = pd.DataFrame({
    'model': [best_ml_model] + formula_models,
    'mse': [best_ml_mse] + formula_mse
}, index=['ML Model', 'Height Formula-Based Model', 'Age Formula-Based Model', 'ID Formula-Based Model'])

df1.to_pickle('table_1.pkl')

# Performing paired ttest
t_stats, p_vals = [], []
for preds in formula_preds:
    residuals_ml = y_test - best_ml_preds
    residuals_formula = y_test - preds
    test_result = stats.ttest_rel(residuals_ml, residuals_formula)
    t_stats.append(test_result.statistic)
    p_vals.append(test_result.pvalue)

# Dataframe for scientific table 2
df2 = pd.DataFrame({
    'model_pair': [f'{best_ml_model} vs {model}' for model in formula_models],
    't_stat': t_stats,
    'p_val': p_vals
}, index=['Pair 1', 'Pair 2', 'Pair 3'])

df2.to_pickle('table_2.pkl')

# SAVE ADDITIONAL RESULTS
# Store additional results in a dictionary for use in the paper.
additional_results = {
 'Number of observations': len(data),
}

with open('additional_results.pkl', 'wb') as f:
    pickle.dump(additional_results, f)

\end{minted}

\subsection{Code Description}

The provided code performs the data analysis for determining the optimal tracheal tube depth (OTTD) in pediatric patients who require mechanical ventilation. The code follows the following steps:

1. Data Loading: The code reads the dataset from the "tracheal\_tube\_insertion.csv" file, which contains information about the patient's sex, age, height, weight, and the OTTD determined by chest X-ray.

2. Preprocessing: The numerical features (age, height, weight) are standardized using the StandardScaler to ensure that they are on the same scale. The sex feature is left unchanged.

3. Machine Learning Models: The code splits the data into training and testing sets using train\_test\_split for model validation. It then evaluates the performance of several machine learning models on predicting the OTTD. The models used include Random Forest, Elastic Net, Support Vector Machines (SVM), and Neural Network. The mean squared error (MSE) is calculated for each model.

4. Formula-based Models: The code also evaluates the performance of three formula-based models for predicting the OTTD based on height, age, and tube ID. The formula-based models calculate the OTTD using simple equations derived from previous studies. The MSE is calculated for each formula-based model.

5. Best Model Selection: The code selects the best-performing machine learning model based on the lowest MSE. The model with the lowest MSE is considered as the best model for predicting the OTTD.

6. Statistical Analysis: The code performs paired t-tests to compare the residuals (the differences between the predicted OTTD and the actual OTTD) of the best machine learning model with each of the formula-based models. The t-statistic and p-values are calculated for each comparison.

7. Results Storage: The code saves the results in two separate files. The first file, "table\_1.pkl", contains the MSE values for the best machine learning model and the three formula-based models, organized in a scientific table format. The second file, "table\_2.pkl", contains the t-statistic and p-values for the paired t-tests between the best machine learning model and each formula-based model. Additionally, the code saves the number of observations in the dataset in the "additional\_results.pkl" file.

The "additional\_results.pkl" file stores additional results in a dictionary format. Currently, the file only contains the number of observations in the dataset. This file can be further modified to include other relevant information or results for use in the research paper.

\subsection{Code Output}

\subsubsection*{table\_1.pkl}

\begin{Verbatim}[tabsize=4]
                                   model        mse
ML Model                      Neural Net   1.173653
Height Formula-Based Model  Height Based  30.139680
Age Formula-Based Model        Age Based   3.682847
ID Formula-Based Model          ID Based   2.344570
\end{Verbatim}

\subsubsection*{table\_2.pkl}

\begin{Verbatim}[tabsize=4]
                        model_pair     t_stat       p_val
Pair 1  Neural Net vs Height Based -70.620160  9.587e-185
Pair 2     Neural Net vs Age Based -16.128130   3.259e-42
Pair 3      Neural Net vs ID Based  18.889568   1.918e-52
\end{Verbatim}

\subsubsection*{additional\_results.pkl}

\begin{Verbatim}[tabsize=4]
{
    'Number of observations': 969,
}
\end{Verbatim}

\section{LaTeX Table Design}
\subsection{{Code}}
The LaTeX Table Design was carried out using the following custom code:

\begin{minted}[linenos, breaklines]{python}

# IMPORT
import pandas as pd
from my_utils import to_latex_with_note, format_p_value, is_str_in_df, split_mapping, AbbrToNameDef

# PREPARATION FOR ALL TABLES
shared_mapping: AbbrToNameDef = {
    'mse': ('Mean Squared Error', 'Performance measure for regression tasks'),
}


# TABLE 1

df = pd.read_pickle('table_1.pkl')

# RENAME ROWS AND COLUMNS
mapping: AbbrToNameDef = shared_mapping.copy()
mapping.update({
    'Age Formula-Based Model': ('Age Formula Measure', None),
    'Height Formula-Based Model': ('Height Formula Measure', None),
    'ID Formula-Based Model': ('Tube ID Formula Measure', None),
    'ML Model': ('Best Machine Learning Model', None),
})

abbrs_to_names, legend = split_mapping(mapping)
df = df.rename(columns=abbrs_to_names, index=abbrs_to_names)

# Save as latex:
to_latex_with_note(df, 'table_1.tex',
                   caption="Comparison of Mean Squared Error between Machine Learning Model and Formula-Based Models", 
                   label='table:comparison_of_mse',
                   note=None,
                   legend=legend)


# TABLE 2

df = pd.read_pickle('table_2.pkl')

# FORMAT VALUES 
df['p_val'] = df['p_val'].apply(format_p_value)

# RENAME ROWS AND COLUMNS
mapping: AbbrToNameDef = {
    'model_pair': ('Comparison of Models', None),
    'Pair 1': ('Comparison Pair 1', None),
    'Pair 2': ('Comparison Pair 2', None),
    'Pair 3': ('Comparison Pair 3', None),
    't_stat': ('T-Statistic', "Measure used in hypothesis testing"),
    'p_val': ('P-value', "Determines the significance of results"),
}

abbrs_to_names, legend = split_mapping(mapping)
df = df.rename(columns=abbrs_to_names, index=abbrs_to_names)

# Save as latex:
to_latex_with_note(df, 'table_2.tex',
                   caption="Paired T-Statistic and P-Value for Comparison of Models",
                   label='table:model_comparison',
                   note=None,
                   legend=legend)

\end{minted}

\subsection{Provided Code}
The code above is using the following provided functions:

\begin{minted}[linenos, breaklines]{python}
def to_latex_with_note(df, filename: str, caption: str, label: str, note: str = None, legend: Dict[str, str] = None, **kwargs):
 """
 Converts a DataFrame to a LaTeX table with optional note and legend added below the table.

 Parameters:
 - df, filename, caption, label: as in `df.to_latex`.
 - note (optional): Additional note below the table.
 - legend (optional): Dictionary mapping abbreviations to full names.
 - **kwargs: Additional arguments for `df.to_latex`.

 Returns:
 - None: Outputs LaTeX file.
 """

def format_p_value(x):
 returns "{:.3g}".format(x) if x >= 1e-06 else "<1e-06"

def is_str_in_df(df: pd.DataFrame, s: str):
 return any(s in level for level in getattr(df.index, 'levels', [df.index]) + getattr(df.columns, 'levels', [df.columns]))

AbbrToNameDef = Dict[Any, Tuple[Optional[str], Optional[str]]]

def split_mapping(abbrs_to_names_and_definitions: AbbrToNameDef):
 abbrs_to_names = {abbr: name for abbr, (name, definition) in abbrs_to_names_and_definitions.items() if name is not None}
 names_to_definitions = {name or abbr: definition for abbr, (name, definition) in abbrs_to_names_and_definitions.items() if definition is not None}
 return abbrs_to_names, names_to_definitions

\end{minted}



\subsection{Code Output}

\subsubsection*{table\_1.tex}

\begin{Verbatim}[tabsize=4]
\begin{table}[h]
\caption{Comparison of Mean Squared Error between Machine Learning Model and
	Formula-Based Models}
\label{table:comparison_of_mse}
\begin{threeparttable}
\renewcommand{\TPTminimum}{\linewidth}
\makebox[\linewidth]{%
\begin{tabular}{llr}
\toprule
 & model & Mean Squared Error \\
\midrule
\textbf{Best Machine Learning Model} & Neural Net & 1.17 \\
\textbf{Height Formula Measure} & Height Based & 30.1 \\
\textbf{Age Formula Measure} & Age Based & 3.68 \\
\textbf{Tube ID Formula Measure} & ID Based & 2.34 \\
\bottomrule
\end{tabular}}
\begin{tablenotes}
\footnotesize
\item \textbf{Mean Squared Error}: Performance measure for regression tasks
\end{tablenotes}
\end{threeparttable}
\end{table}

\end{Verbatim}

\subsubsection*{table\_2.tex}

\begin{Verbatim}[tabsize=4]
\begin{table}[h]
\caption{Paired T-Statistic and P-Value for Comparison of Models}
\label{table:model_comparison}
\begin{threeparttable}
\renewcommand{\TPTminimum}{\linewidth}
\makebox[\linewidth]{%
\begin{tabular}{llrl}
\toprule
 & Comparison of Models & T-Statistic & P-value \\
\midrule
\textbf{Comparison Pair 1} & Neural Net vs Height Based & -70.6 & $<$1e-06 \\
\textbf{Comparison Pair 2} & Neural Net vs Age Based & -16.1 & $<$1e-06 \\
\textbf{Comparison Pair 3} & Neural Net vs ID Based & 18.9 & $<$1e-06 \\
\bottomrule
\end{tabular}}
\begin{tablenotes}
\footnotesize
\item \textbf{T-Statistic}: Measure used in hypothesis testing
\item \textbf{P-value}: Determines the significance of results
\end{tablenotes}
\end{threeparttable}
\end{table}

\end{Verbatim}


\bibliographystyle{unsrt}
\bibliography{citations}

\end{document}
