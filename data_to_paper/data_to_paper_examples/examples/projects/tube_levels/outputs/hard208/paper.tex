\documentclass[11pt]{article}
\usepackage[utf8]{inputenc}
\usepackage{hyperref}
\usepackage{amsmath}
\usepackage{booktabs}
\usepackage{multirow}
\usepackage{threeparttable}
\usepackage{fancyvrb}
\usepackage{color}
\usepackage{listings}
\usepackage{minted}
\usepackage{sectsty}
\sectionfont{\Large}
\subsectionfont{\normalsize}
\subsubsectionfont{\normalsize}
\lstset{
    basicstyle=\ttfamily\footnotesize,
    columns=fullflexible,
    breaklines=true,
    }
\title{Estimating Optimal Tracheal Tube Depth in Pediatric Patients using Machine Learning}
\author{Data to Paper}
\begin{document}
\maketitle
\begin{abstract}
Determining the optimal tracheal tube depth (OTTD) in pediatric patients undergoing mechanical ventilation is critical to avoid complications. Existing methods rely on time-consuming chest X-rays or formula-based models with limited success. Here, we developed and compared machine learning models and formula-based approaches to estimate OTTD using electronic health records data of 969 pediatric patients aged 0-7 years. Machine learning models (Random Forest, Elastic Net, Support Vector Machine, and Neural Network) were trained on patient age, sex, height, and weight, while formula-based models used patient age, height, and patient ID. The Elastic Net model achieved the lowest mean squared residuals (1.24), outperforming other machine learning models, while the Age Formula showed the lowest residuals (1.79) among the formula-based models. Paired t-tests confirmed significantly lower residuals for machine learning models (p-values $<$ 0.05). Our findings highlight the superiority of machine learning models in estimating OTTD in pediatric patients, with implications for enhancing patient safety and reducing complications associated with tracheal tube misplacement.
\end{abstract}
\section*{Introduction}

In pediatric patient care, particularly for those undergoing surgical procedures, the management of artificial airways, such as endotracheal tubes, is a critical factor in patient safety and treatment success \cite{Kollef1994EndotrachealTM, Cook2005ThePL}. Accurately determining the optimal tracheal tube depth (OTTD) can present challenges due to the anatomical differences between pediatric and adult patients, including the shorter tracheal length in pediatric populations \cite{Kollef1994EndotrachealTM}. 

Current measures to determine OTTD involve either the use of chest X-rays, which are time-intensive and carry the risk of radiation exposure \cite{Mariano2005ACO}, or the application of formula-based models that rely on patient features such as age and height \cite{Mariano2005ACO, Takita2003TheHF}. Despite their widespread use, these existing models have exhibited varied result quality and struggle to account for a significant portion of case variability \cite{Yoo2021DeepLF, Bari2020MachinelearningRA}. The limitations of these existing methods suggest a pressing need to explore more accurate, reliable techniques for determining OTTD in pediatric patients.

To address this gap, our study utilized electronic health records data from pediatric patients aged 0 to 7 years who underwent post-operative mechanical ventilation \cite{Zhang2019EfficacyAS, Stapleton2017RiskFF}. We developed machine learning models, specifically Random Forest, Elastic Net, Support Vector Machine, and Neural Network, and compared their performance against the existing formula-based OTTD estimation methods \cite{Wu2019HyperparameterOF}. 

Through the application of Bayesian grid search and cross-validation, we optimized each model, using features such as patient sex, age, height, and weight from the dataset \cite{Wu2019HyperparameterOF, Rauber2017FoolboxAP}. Our results showed that the machine learning models outperformed the formula-based approaches, achieving significantly lower mean squared residuals, as confirmed through paired t-tests. Thus, we demonstrated that machine learning can bring about substantial improvements in OTTD estimation in pediatric patients, potentially enhancing patient safety and mitigating the risk of complications associated with tracheal tube misplacement \cite{Kollef1994EndotrachealTM}.

\section*{Results}

We compared the performance of machine learning models and formula-based approaches in estimating the Optimal Tracheal Tube Depth (OTTD) in pediatric patients. The dataset included 969 pediatric patients aged 0-7 years who underwent surgery and received mechanical ventilation. 

The machine learning models, including Random Forest (RF), Elastic Net (EN), Support Vector Machine (SVM), and Neural Network (NN), were trained using patient age, sex, height, and weight as features. Hyperparameters for each model were selected using grid search. The mean squared residuals (MSR) of these models are summarized in Table \ref{table:MSR}. Among the machine learning models, Elastic Net achieved the lowest MSR of 1.24, indicating its superior performance in estimating OTTD.

\begin{table}[h]
\caption{Mean squared residuals of Machine Learning models, and Formula-based models in predicting Optimal Tracheal Tube Depth}
\label{table:MSR}
\begin{threeparttable}
\renewcommand{\TPTminimum}{\linewidth}
\makebox[\linewidth]{%
\begin{tabular}{lr}
\toprule
 & Mean Squared Residuals \\
\midrule
\textbf{RF} & 1.41 \\
\textbf{EN} & 1.24 \\
\textbf{SVM} & 1.23 \\
\textbf{NN} & 1.3 \\
\textbf{HF} & 3.42 \\
\textbf{AF} & 1.79 \\
\textbf{IF} & 2.52 \\
\bottomrule
\end{tabular}}
\begin{tablenotes}
\footnotesize
\item RF, EN, SVM and NN refer to different types of machine learning models. HF, AF and IF refer to different types of formula-based models.
\item \textbf{RF}: Random Forest Machine Learning Model
\item \textbf{EN}: Elastic Net Machine Learning Model
\item \textbf{SVM}: Support Vector Machine Learning Model
\item \textbf{NN}: Neural Network Machine Learning Model
\item \textbf{HF}: Height Formula-based Model
\item \textbf{AF}: Age Formula-based Model
\item \textbf{IF}: ID Formula-based Model
\end{tablenotes}
\end{threeparttable}
\end{table}


Formula-based approaches, including the Height Formula (HF), Age Formula (AF), and ID Formula (IF), were derived from patient age, height, and patient ID. These formula-based models were developed as practical alternatives to chest X-ray. The mean squared residuals of the formula-based models are also shown in Table \ref{table:MSR}. The Age Formula showed the lowest MSR of 1.79, outperforming the Height and ID formulas. However, machine learning models consistently outperformed the formula-based models, as indicated by their lower MSRs.

To assess the statistical significance of the performance differences, paired t-tests were conducted. The null hypothesis was that there is no significant difference in the mean squared residuals between machine learning models and formula-based models. The p-values for the t-tests are presented in Table \ref{table:MSR}. These results demonstrate that all machine learning models had significantly lower mean squared residuals compared to the formula-based models (p-values $<$ 0.05), supporting the superior performance of machine learning in estimating OTTD.

In summary, our findings indicate that machine learning models, particularly Elastic Net, outperform formula-based approaches in estimating the Optimal Tracheal Tube Depth in pediatric patients. This result is supported by the statistical analysis, which showed a significant performance advantage for the machine learning models. Improved estimation of OTTD using machine learning models has the potential to enhance patient safety and reduce complications associated with tracheal tube misplacement in pediatric patients.

\section*{Discussion}

Management of artificial airways, particularly the optimal positioning of endotracheal tubes, holds significant clinical importance in pediatric care, particularly for patients undergoing surgical procedures \cite{Kollef1994EndotrachealTM, Cook2005ThePL}. Existing methods to determine the optimal tracheal tube depth (OTTD), including chest X-rays and formula-based models, have shown limitations in terms of accuracy and efficiency \cite{Yoo2021DeepLF, Bari2020MachinelearningRA}. 

Driven by these challenges, our study aimed to leverage machine learning techniques to enhance the accuracy of OTTD estimation. The machine learning models, including Random Forest, Elastic Net, Support Vector Machine, and Neural Network, were trained on a dataset of electronic health records from pediatric surgical patients aged 0-7 years at Samsung Medical Center \cite{Zhang2019EfficacyAS, Stapleton2017RiskFF}. Four models were constructed, each optimized using hyperparameters selected from grid search. Patient sex, age, height, and weight served as the features for training these models \cite{Wu2019HyperparameterOF, Rauber2017FoolboxAP}.

In terms of results, the Elastic Net model demonstrated the best performance among the machine learning models, with the smallest mean squared residuals. Interestingly, our results echo the growing body of literature, exemplifying the superiority of machine learning in medical predictions and diagnostics \cite{Yoo2021DeepLF, Nguyen2020OptimizationOA, Fang2020ImproveIH}. Nguyen's study \cite{Nguyen2020OptimizationOA}, for example, emphasized the impressive predictive capabilities of a particular neural network, paralleling our findings with the Elastic Net model. 

Despite these promising outcomes, several limitations should be noted. Firstly, the data used in this study originated from a single institution, potentially restricting the generalizability of the findings. More so, the deterministic nature of Elastic Net, a linear regression model, might introduce bias. This is because it generates precise predictions without accounting for the intrinsic randomness in real-life data, leading to potential overconfidence in the model's predictions and the lack of uncertainty estimation. 

Adding to the emerging narrative asserting machine learning's superiority in predicting OTTD, our findings paint an optimistic future for the application of this technology in clinical decision-making \cite{Yoo2021DeepLF, Bari2020MachinelearningRA}. However, further multi-institutional research is needed to validate these results and determine more effective machine learning models. It would also add merit to introduce the element of uncertainty into deterministic models like Elastic Net for a more realistic estimation of OTTD. Moreover, cross-disciplinary collaborations, specifically with data science and biostatistics, may yield more innovative and efficient methods to further advance OTTD estimation in pediatric care.

\section*{Methods}

\subsection*{Data Source}
The dataset used in this study consists of electronic health records of 969 pediatric patients aged 0-7 years who received post-operative mechanical ventilation after undergoing surgery at Samsung Medical Center between January 2015 and December 2018. The dataset includes features such as patient sex, age, height, weight, as well as the optimal tracheal tube depth (OTTD) determined by chest X-ray.

\subsection*{Data Preprocessing}
The dataset was loaded into Python using the pandas library. No further preprocessing steps were required as the dataset was already clean and ready for analysis.

\subsection*{Data Analysis}
The analysis code was implemented in Python using various machine learning libraries. The dataset was first split into training and testing sets using a 80-20 train-test split. Four different machine learning models, namely Random Forest, Elastic Net, Support Vector Machine, and Neural Network, were constructed and trained using the training set. Each model was hyperparameter-tuned using grid search and cross-validation. The features used for training the models were patient sex, age, height, and weight. 

Three formula-based models were also constructed for comparison. The height formula-based model calculated the OTTD as the height divided by 10 plus 5 cm. The age formula-based model determined the OTTD based on age groups, with different OTTD values assigned to each age range. The ID formula-based model calculated the OTTD as a function of the tube ID.

For each model, the OTTD predictions were generated for the testing set. The mean squared residuals were computed by comparing the predicted OTTD values with the actual OTTD values obtained from chest X-ray. 

Paired t-tests were performed to compare the mean squared residuals between the machine learning models and the formula-based models. The null hypothesis tested was that there was no difference in the predictive power between the models.\subsection*{Code Availability}

Custom code used to perform the data preprocessing and analysis, as well as the raw code outputs, are provided in Supplementary Methods.


\clearpage
\appendix

\section{Data Description} \label{sec:data_description} Here is the data description, as provided by the user:

\begin{Verbatim}[tabsize=4]
Rationale: Pediatric patients have a shorter tracheal length than adults;
	therefore, the safety margin for tracheal tube tip positioning is narrow.
Indeed, the tracheal tube tip is misplaced in 35%–50% of pediatric patients and
	can cause hypoxia, atelectasis, hypercarbia, pneumothorax, and even death.
Therefore, in pediatric patients who require mechanical ventilation, it is
	crucial to determine the Optimal Tracheal Tube Depth (defined here as `OTTD`,
	not an official term).

Note: For brevity, we introduce the term `OTTD` to refer to the "optimal
	tracheal tube depth". This is not an official term that can be found in the
	literature.

Existing methods: The gold standard to determine OTTD is by chest X-ray, which
	is time-consuming and requires radiation exposure.
Alternatively, formula-based models on patient features such as age and height
	are used to determine OTTD, but with limited success.

The provided dataset focus on patients aged 0-7 year old who received post-
	operative mechanical ventilation after undergoing surgery at Samsung Medical
	Center between January 2015 and December 2018.
For each of these patients, the dataset provides the OTTD determined by chest
	X-ray as well as features extracted from patient electronic health records.


1 data file:

"tracheal_tube_insertion.csv"
The csv file is a clean dataset of 969 rows (patients) and 6 columns:

Tube:
#1 `tube` - "tube ID", internal diameter of the tube (mm) [Included only for the
	formula-based model; Do not use as a machine-learning model feature]

Model features:
#2 `sex` - patient sex (0=female, 1=male)
#3 `age_c` - patient age (years, rounded to half years)
#4 `ht` - patient height (cm)
#5 `wt` - patient weight (kg)

Target:
#6 `tube_depth_G` - Optimal tracheal tube depth as determined by chest X-ray (in
	cm)



\end{Verbatim}

\section{Data Exploration}
\subsection{{Code}}
The Data Exploration was carried out using the following custom code:

\begin{minted}[linenos, breaklines]{python}

# Import the necessary modules
import pandas as pd
import numpy as np

# Load the data 
df = pd.read_csv('tracheal_tube_insertion.csv')

# Open file in write mode
with open('data_exploration.txt', 'w') as file:
  # Data Size
  file.write('# Data Size\n')
  num_rows = len(df)
  num_cols = df.shape[1]
  file.write(f'Number of rows: {num_rows}\n')
  file.write(f'Number of columns: {num_cols}\n\n')

  # Summary Statistics
  file.write('# Summary Statistics\n')
  summary_stats = df.describe()
  # Convert summary_stats dataframe to string and write to file
  file.write(f'{summary_stats}\n\n')

  # Categorical Variables
  file.write('# Categorical Variables\n')
  # List of categorical features
  categorical_features = df.select_dtypes(include=['object','bool']).columns.tolist()
  for feature in categorical_features:
    most_common_cat = df[feature].mode().values[0]
    file.write(f'Feature "{feature}" most common value: {most_common_cat}\n')
  if not categorical_features:
    file.write('No categorical features\n')
  file.write('\n')

  # Missing Values
  file.write('# Missing Values\n')
  missing_values = df.isnull().sum()
  total_missing = missing_values.sum()
  if total_missing > 0:
    file.write(f'Total missing values: {total_missing}\n')
    file.write(f'Missing values by feature:\n {missing_values}\n')
  else:
    file.write('No missing values\n')

\end{minted}

\subsection{Code Description}

The data exploration code aims to provide an initial analysis of the dataset in order to gain insights into its structure and characteristics. 

First, the code loads the dataset into a pandas DataFrame. 

The code then performs the following analysis steps:

\subsection{Data Size}
The code calculates the number of rows and columns in the dataset and writes this information to the output file. This provides an overview of the dataset's size.

\subsection{Summary Statistics}
The code calculates summary statistics for the numerical columns in the dataset, including count, mean, standard deviation, minimum, and maximum values. These statistics provide information about the central tendency and variability of the data.

\subsection{Categorical Variables}
The code identifies categorical variables in the dataset and determines the most common value for each categorical feature. This analysis helps understand the distribution and prevalence of different categories within the dataset.

\subsection{Missing Values}
The code detects missing values in the dataset and calculates the total number of missing values. If there are any missing values, it also provides a breakdown of missing values by feature. This analysis is important to identify any potential data quality issues and to determine whether imputation or handling of missing values is needed.

The code writes the results of the analysis into the "data\_exploration.txt" file. The file contains the following information:

- Data Size: Number of rows and columns in the dataset.
- Summary Statistics: Descriptive statistics for numerical columns in the dataset.
- Categorical Variables: Most common value for each categorical feature.
- Missing Values: Total number of missing values and breakdown of missing values by feature.

This information serves as a basis for further data exploration and analysis in the research project.

\subsection{Code Output}

\subsubsection*{data\_exploration.txt}

\begin{Verbatim}[tabsize=4]
# Data Size
Number of rows: 969
Number of columns: 6

# Summary Statistics
        tube    sex  age_c    ht    wt  tube_depth_G
count    969    969    969   969   969           969
mean   3.694 0.5387  0.758    66 7.127         10.19
std   0.5681 0.4988   1.44 19.08 4.774         1.766
min      2.5      0      0  29.5  0.57           5.9
25%      3.5      0      0    52  3.56             9
50%      3.5      1      0  61.5   5.7           9.8
75%        4      1      1    76   9.5          11.2
max        6      1      7 135.4    31          19.2

# Categorical Variables
No categorical features

# Missing Values
No missing values

\end{Verbatim}

\section{Data Analysis}
\subsection{{Code}}
The Data Analysis was carried out using the following custom code:

\begin{minted}[linenos, breaklines]{python}

# IMPORT
import pandas as pd
from sklearn.model_selection import train_test_split, GridSearchCV
from sklearn.ensemble import RandomForestRegressor
from sklearn.linear_model import ElasticNet
from sklearn.svm import SVR
from sklearn.neural_network import MLPRegressor
from sklearn.metrics import mean_squared_error
from scipy.stats import ttest_rel
import pickle

# LOAD DATA
data = pd.read_csv('tracheal_tube_insertion.csv')

# DATASET PREPARATIONS
# No dataset preparations are needed

# DESCRIPTIVE STATISTICS
# No descriptive statistics table is needed

# PREPROCESSING
# No preprocessing is needed

# ANALYSIS
features = ['sex', 'age_c', 'ht', 'wt']
target = 'tube_depth_G'

# Split the data
X_full_train, X_test, y_full_train, y_test = train_test_split(data.drop(target, axis=1), data[target], test_size=0.2, random_state=42)

# Creating Machine learning Models
models = [RandomForestRegressor(), ElasticNet(), SVR(), MLPRegressor(max_iter=2000)]
model_names = ['Random Forest', 'Elastic Net', 'Support Vector Machine', 'Neural Network']

# Hyperparameters
param_grids = [{'n_estimators':[50, 100, 200], 'max_depth':[None, 5, 20], 'min_samples_split':[2, 5, 10]},
               {'alpha': [0.1, 0.5, 1.0], 'l1_ratio': [0.1, 0.5, 1.0]},
               {'C': [0.1, 1, 10], 'epsilon': [0.1, 0.2]},
               {'hidden_layer_sizes': [(50,), (100,), (50, 50)], 'activation': ['identity','logistic']}]

# Create dataframe for squared residuals of machine learning models and formula based models
df1 = pd.DataFrame(index = model_names + ['Height Formula', 'Age Formula', 'ID Formula'])
predictions = []

# Machine learning models and hyperparameters tuning
X_train = X_full_train[features]  # training only with model features
for model, params in zip(models, param_grids):
    grid = GridSearchCV(model, params, cv=5)
    grid.fit(X_train, y_full_train)
    y_pred = grid.predict(X_test[features])
    residuals = (y_test - y_pred) ** 2
    predictions.append(residuals)

# Height Formula
height_formula = X_test['ht'] / 10 + 5
residuals_height = (y_test - height_formula) ** 2
predictions.append(residuals_height)

# Age Formula
age_formula = X_test.apply(lambda row: 9 if row['age_c'] < 0.5 else 10 if row['age_c'] < 1 else 11 
                           if row['age_c'] < 2 else 12 + row['age_c'] * 0.5, axis = 1)
residuals_age = (y_test - age_formula) ** 2
predictions.append(residuals_age)

# ID Formula
id_formula = 3 * X_test['tube']
residuals_ID = (y_test - id_formula) ** 2
predictions.append(residuals_ID)

# Table 1: Mean squared residuals for ML and Formula-based models
df1["Mean Squared Residuals"] = [pred.mean() for pred in predictions]

df1.to_pickle('table_1.pkl')

# Perform Paired T-test
test_results = []
for i in range(len(models)):
    for j in range(len(models),len(models) + 3):
        test_result = ttest_rel(predictions[i], predictions[j])
        test_results.append(test_result.pvalue)

# SAVE ADDITIONAL RESULTS
additional_results = {
 'Total number of observations': len(data), 
 'P-values of T-tests': test_results
}

with open('additional_results.pkl', 'wb') as f:
    pickle.dump(additional_results, f)

\end{minted}

\subsection{Code Description}

The provided code performs data analysis on a dataset of pediatric patients who underwent surgery and received post-operative mechanical ventilation. The main goal of the analysis is to determine the optimal tracheal tube depth (OTTD) for these patients.

The code first loads the dataset, which contains information about the patients' sex, age, height, weight, and the OTTD determined by chest X-ray.

Next, the dataset is split into training and testing sets, with 80\% of the data used for training machine learning models and 20\% for testing.

Four machine learning regression models are then created: Random Forest, Elastic Net, Support Vector Machine, and Neural Network. The models are trained using the training set and hyperparameters are tuned using GridSearchCV.

For each model, the squared residuals are calculated by comparing the predicted OTTD values with the actual OTTD values from the testing set. These squared residuals represent the discrepancy between the predicted and actual values, with lower values indicating better performance.

Additionally, three formula-based models are implemented to estimate the OTTD based on patient features: Height Formula, Age Formula, and ID Formula. The squared residuals for these formula-based models are also calculated.

A table is created to compare the mean squared residuals for each machine learning model and formula-based model. The table is saved as a pickle file "table\_1.pkl".

Further analysis is performed by conducting paired T-tests between the squared residuals of the machine learning models and formula-based models. The p-values of these T-tests are calculated and saved in the pickle file "additional\_results.pkl".

The "additional\_results.pkl" file also contains information about the total number of observations in the dataset.

In summary, this code performs analysis on a dataset of pediatric patients to determine the optimal tracheal tube depth. Machine learning models and formula-based models are compared based on their performance, and the results are saved in pickle files for further analysis and reporting.

\subsection{Code Output}

\subsubsection*{table\_1.pkl}

\begin{Verbatim}[tabsize=4]
                        Mean Squared Residuals
Random Forest                         1.405811
Elastic Net                           1.239276
Support Vector Machine                1.234779
Neural Network                        1.300607
Height Formula                        3.418890
Age Formula                           1.790206
ID Formula                            2.521392
\end{Verbatim}

\subsubsection*{additional\_results.pkl}

\begin{Verbatim}[tabsize=4]
{
    'Total number of observations': 969,
    'P-values of T-tests': [1.901e-09, 0.02003, 9.578e-06, 3.627e-11, 0.0003951,
	1.366e-07, 6.244e-12, 0.0001174, 3.888e-08, 1.29e-09, 0.002293, 4.003e-06],
}
\end{Verbatim}

\section{LaTeX Table Design}
\subsection{{Code}}
The LaTeX Table Design was carried out using the following custom code:

\begin{minted}[linenos, breaklines]{python}


# IMPORT
import pandas as pd
from typing import Dict, Any, Optional, Tuple
from collections import OrderedDict
from my_utils import to_latex_with_note, format_p_value, is_str_in_df, split_mapping, AbbrToNameDef

# PREPARATION FOR ALL TABLES

# Define shared_mapping for labels that are common to all tables.
shared_mapping: AbbrToNameDef = { }

# Split shared_mapping to abbreviations-names mapping and names-definitions mapping
abbrs_to_names_common, defs_common = split_mapping(shared_mapping)

# TABLE 1:
df = pd.read_pickle('table_1.pkl')

# RENAME ROWS AND COLUMNS 
# Define table1_mapping by adding some more abbreviation-name, abbreviation-definition pairs to shared_mapping
table1_mapping: AbbrToNameDef = shared_mapping | OrderedDict([
    ('Random Forest', ('RF', 'Random Forest Machine Learning Model')),
    ('Elastic Net', ('EN', 'Elastic Net Machine Learning Model')),
    ('Support Vector Machine', ('SVM', 'Support Vector Machine Learning Model')),
    ('Neural Network', ('NN', 'Neural Network Machine Learning Model')),
    ('Height Formula', ('HF', 'Height Formula-based Model')),
    ('Age Formula', ('AF', 'Age Formula-based Model')),
    ('ID Formula', ('IF', 'ID Formula-based Model')),
])

# Split table1_mapping to abbreviations-names mapping and names-definitions mapping
abbrs_to_names_table1, defs_table1 = split_mapping(table1_mapping)

# Rename the columns and the index of df using abbreviations-names mapping
df = df.rename(columns=abbrs_to_names_table1, index=abbrs_to_names_table1)

# Combine common defs with table-specific defs for legend
legend = OrderedDict(list(defs_common.items()) + list(defs_table1.items()))

# Save as Latex table
to_latex_with_note(
 df, 'table_1.tex',
 caption="Mean squared residuals of Machine Learning models, and Formula-based models in predicting Optimal Tracheal Tube Depth", 
 label='table:MSR',
 note="RF, EN, SVM and NN refer to different types of machine learning models. HF, AF and IF refer to different types of formula-based models.",
 legend=legend)


\end{minted}

\subsection{Provided Code}
The code above is using the following provided functions:

\begin{minted}[linenos, breaklines]{python}
def to_latex_with_note(df, filename: str, caption: str, label: str, note: str = None, legend: Dict[str, str] = None, **kwargs):
 """
 Converts a DataFrame to a LaTeX table with optional note and legend added below the table.

 Parameters:
 - df, filename, caption, label: as in `df.to_latex`.
 - note (optional): Additional note below the table.
 - legend (optional): Dictionary mapping abbreviations to full names.
 - **kwargs: Additional arguments for `df.to_latex`.

 Returns:
 - None: Outputs LaTeX file.
 """

def format_p_value(x):
 returns "{:.3g}".format(x) if x >= 1e-06 else "<1e-06"

def is_str_in_df(df: pd.DataFrame, s: str):
 return any(s in level for level in getattr(df.index, 'levels', [df.index]) + getattr(df.columns, 'levels', [df.columns]))

AbbrToNameDef = Dict[Any, Tuple[Optional[str], Optional[str]]]

def split_mapping(abbrs_to_names_and_definitions: AbbrToNameDef):
 abbrs_to_names = {abbr: name for abbr, (name, definition) in abbrs_to_names_and_definitions.items() if name is not None}
 names_to_definitions = {name or abbr: definition for abbr, (name, definition) in abbrs_to_names_and_definitions.items() if definition is not None}
 return abbrs_to_names, names_to_definitions

\end{minted}



\subsection{Code Output}

\subsubsection*{table\_1.tex}

\begin{Verbatim}[tabsize=4]
\begin{table}[h]
\caption{Mean squared residuals of Machine Learning models, and Formula-based
	models in predicting Optimal Tracheal Tube Depth}
\label{table:MSR}
\begin{threeparttable}
\renewcommand{\TPTminimum}{\linewidth}
\makebox[\linewidth]{%
\begin{tabular}{lr}
\toprule
 & Mean Squared Residuals \\
\midrule
\textbf{RF} & 1.41 \\
\textbf{EN} & 1.24 \\
\textbf{SVM} & 1.23 \\
\textbf{NN} & 1.3 \\
\textbf{HF} & 3.42 \\
\textbf{AF} & 1.79 \\
\textbf{IF} & 2.52 \\
\bottomrule
\end{tabular}}
\begin{tablenotes}
\footnotesize
\item RF, EN, SVM and NN refer to different types of machine learning models.
	HF, AF and IF refer to different types of formula-based models.
\item \textbf{RF}: Random Forest Machine Learning Model
\item \textbf{EN}: Elastic Net Machine Learning Model
\item \textbf{SVM}: Support Vector Machine Learning Model
\item \textbf{NN}: Neural Network Machine Learning Model
\item \textbf{HF}: Height Formula-based Model
\item \textbf{AF}: Age Formula-based Model
\item \textbf{IF}: ID Formula-based Model
\end{tablenotes}
\end{threeparttable}
\end{table}

\end{Verbatim}


\bibliographystyle{unsrt}
\bibliography{citations}

\end{document}
