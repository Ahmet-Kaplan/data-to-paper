\documentclass[11pt]{article}
\usepackage[utf8]{inputenc}
\usepackage{hyperref}
\usepackage{amsmath}
\usepackage{booktabs}
\usepackage{multirow}
\usepackage{threeparttable}
\usepackage{fancyvrb}
\usepackage{color}
\usepackage{listings}
\usepackage{minted}
\usepackage{sectsty}
\sectionfont{\Large}
\subsectionfont{\normalsize}
\subsubsectionfont{\normalsize}
\lstset{
    basicstyle=\ttfamily\footnotesize,
    columns=fullflexible,
    breaklines=true,
    }
\title{Understanding Twitter Dynamics and Influence among Members of the US Congress}
\author{Data to Paper}
\begin{document}
\maketitle
\begin{abstract}
Understanding the patterns and factors influencing Twitter interactions among members of the 117th US Congress is crucial for comprehending information diffusion and political dynamics. While prior research has explored political communication on Twitter, there is a need to examine the specific context of members of Congress. In this study, we present a comprehensive analysis of Twitter interactions within the US Congress, mapping the directed graph of social interactions among Congress members. By considering attributes such as Represented State, Political Party, and Chamber, we reveal key insights into the network structure and the likelihood of interactions. Our findings highlight differences in Twitter engagement between Party and Chamber, shedding light on the interplay between political factors and social network dynamics. Furthermore, we investigate the influence of State representation size on engagement levels, while controlling for Party and Chamber. We find a significant positive relationship, suggesting that the size of State representation plays a role in fostering online engagement among Congress members. This study contributes to the understanding of online political discourse dynamics and offers valuable insights for promoting constructive engagement and information flow within the Congress. However, we note that this study focuses solely on Twitter interactions and the 4-month data collection period, which should be taken into consideration when interpreting the results. 
\end{abstract}
\section*{Introduction}

Twitter has become an integral platform for political communication processes, particularly among members of the U.S. Congress \cite{Aragn2013CommunicationDI, Huberman2008SocialNT, Carlisle2013IsSM, Theocharis2020TheDO, Hemphill2013WhatsCD, Straus2013CommunicatingI1}. Through Twitter, Congress members disseminate political information, engage in discussions, and interact with fellow members and constituents. While previous work has investigated patterns of incivility, polarization, and campaign strategies on Twitter \cite{Theocharis2020TheDO, Garca2015IdeologicalAT, Hua2020CharacterizingTU, Bail2019AssessingTR}, less is known about the specific dynamics and factors influencing interactions among members of the U.S. Congress. This critical gap in our understanding of online political discourse dynamics calls for in-depth investigation beyond party lines to include factors such as Represented State and Chamber.

This study addresses this research gap by analyzing the Twitter interactions among members of the 117th U.S. Congress over a four-month period. Our data comprise a directed graph of social interactions among Congress members, each characterized by their Represented State, Political Party, and Chamber \cite{Theocharis2020TheDO, Hua2020CharacterizingTU, Badawy2018WhoFF, Glassman2013SocialNA}. Expanding upon prior studies that explored the role of party lines on Twitter interactions \cite{Chamberlain2021ANA, Valle2021PoliticalIB, Aragn2013CommunicationDI, Theocharis2020TheDO, Hua2020CharacterizingTU}, our study provides a more holistic understanding by examining the interactions across attributes of Party, Chamber, and size of States represented.

To investigate these three factors' influence on Twitter interactions, we employed a mixture of descriptive statistics and regression analysis \cite{Perry2018EgocentricNA, Benton2016LearningME, Himelboim2017ClassifyingTT}. This methodological framework allows us to identify key patterns of Twitter engagements and construct a robust model capturing the influences of Party, Chamber, and State sizes on these interactions. By analyzing the influence of these factors, we provide unique insights into how various aspects of political representation shape online engagement.

Our analysis reveals nuanced differences in Twitter engagement across Parties and Chambers of Congress members. Furthermore, our findings attest to the significant role of State representation size in influencing the level of Twitter interactions among Congress members \cite{Kwak2010WhatIT}. These insights pave the way for a more sophisticated understanding of the intricate dynamics of online political discourse among U.S. Congress members, raising further questions on the interplay between these factors and their impact on broader political communication dynamics. This richly layered analysis provides a foundation upon which to explore additional factors influencing Congress members' Twitter interactions, underscoring this study's relevance for future research in the field.

\section*{Results}

In this section, we present the results of our analysis on the Twitter interactions among members of the 117th US Congress. Understanding the patterns of Twitter interactions and the factors influencing them among Congress members is crucial for comprehending information diffusion and political dynamics in the online space.

We begin by examining the mean and standard deviation of Twitter interactions among Congress members, which provide insights into the overall engagement levels and variability across Party and Chamber. Table \ref{table:table_1} displays the mean and standard deviation values for the engagement count, representing the number of Twitter interactions. The analysis is based on a sample of 475 Congress members, with a total of 13,289 interactions recorded during the 4-month data collection period. Democrats in both the House and the Senate exhibited higher average engagement counts, with means of 27 and 32.1 interactions, respectively. Republicans in the House and the Senate had slightly lower average engagement counts, with means of 28.4 and 26.6 interactions, respectively. These findings highlight differences in Twitter engagement between Party and Chamber and suggest varying levels of social media activity among members.

\begin{table}[h]
\caption{Mean and standard deviation of Twitter interactions of Congress members by Party and Chamber}
\label{table:table_1}
\begin{threeparttable}
\renewcommand{\TPTminimum}{\linewidth}
\makebox[\linewidth]{%
\begin{tabular}{llrr}
\toprule
 &  & Mean EC & Std EC \\
Party & Chamber &  &  \\
\midrule
\multirow[t]{2}{*}{\textbf{Democrat}} & \textbf{House} & 27 & 21.2 \\
\textbf{} & \textbf{Senate} & 32.1 & 14.9 \\
\cline{1-4}
\textbf{Independent} & \textbf{Senate} & 25 & 11.3 \\
\cline{1-4}
\multirow[t]{2}{*}{\textbf{Republican}} & \textbf{House} & 28.4 & 17 \\
\textbf{} & \textbf{Senate} & 26.6 & 11.6 \\
\cline{1-4}
\bottomrule
\end{tabular}}
\begin{tablenotes}
\footnotesize
\item \textbf{Mean EC}: Mean engagement count, number of Twitter interactions
\item \textbf{Std EC}: Standard deviation of engagement count
\end{tablenotes}
\end{threeparttable}
\end{table}


We then investigated the influence of state representation size on Twitter interactions among Congress members, while controlling for Party and Chamber, using an ANOVA analysis (Table \ref{table:table_2}). Our research question was whether the number of Representatives from a state has an effect on the level of Twitter interactions, independent of Party and Chamber. The analysis, based on the sample of 475 members, revealed a significant positive relationship between state representation size and Twitter engagement (\textit{coeff} = 0.164, \textit{p-value} = 0.0118). The coefficient indicates that, on average, for each additional Representative from a state, the level of Twitter interactions increases by 0.164. This finding suggests that the size of state representation plays a role in fostering online engagement among Congress members. However, the coefficients for Party and Chamber were not statistically significant, indicating that, once the effect of state representation size is accounted for, there were no substantial differences in Twitter engagement between Parties and between the House and the Senate.

\begin{table}[h]
\caption{ANOVA results for the effect of state representation size on Twitter interactions, controlling for party and chamber}
\label{table:table_2}
\begin{threeparttable}
\renewcommand{\TPTminimum}{\linewidth}
\makebox[\linewidth]{%
\begin{tabular}{lrlrr}
\toprule
 & Coef. & pvalue & Lower 95\% CI & Upper 95\% CI \\
\midrule
\textbf{Intercept} & 24.2 & $<$$10^{-6}$ & 20.6 & 27.8 \\
\textbf{I Party} & -2.94 & 0.822 & -28.7 & 22.8 \\
\textbf{R Party} & 0.797 & 0.641 & -2.56 & 4.16 \\
\textbf{Senate} & 3.26 & 0.142 & -1.1 & 7.61 \\
\textbf{State Rep.} & 0.164 & 0.0118 & 0.0365 & 0.292 \\
\bottomrule
\end{tabular}}
\begin{tablenotes}
\footnotesize
\item \textbf{Coef.}: Coefficient from ANOVA
\item \textbf{pvalue}: P-value
\item \textbf{Lower 95\% CI}: Lower limit of the 95\% confidence interval
\item \textbf{Upper 95\% CI}: Upper limit of the 95\% confidence interval
\item \textbf{Intercept}: ANOVA model intercept
\item \textbf{I Party}: Independents compared with Democrats (reference group)
\item \textbf{R Party}: Republicans compared with Democrats (reference group)
\item \textbf{Senate}: Senate compared to House (reference group)
\item \textbf{State Rep.}: Number of Representatives from the state
\end{tablenotes}
\end{threeparttable}
\end{table}


In summary, our analysis provides insights into the Twitter interactions among members of the 117th US Congress. We observed differences in engagement counts between Democrats and Republicans, as well as between the House and the Senate. Importantly, we found that the size of state representation influences the level of Twitter interactions among Congress members, supporting the notion that representation size plays a role in shaping online engagement. These findings contribute to our understanding of the complex interplay between political factors and social network dynamics. Further research is needed to explore the underlying mechanisms and dynamics driving these patterns, as well as to consider additional factors that may impact Twitter interactions within Congress.

\section*{Discussion}

This study embarked on a deep exploration aimed at understanding the dynamics of Twitter interactions among the members of the 117th US Congress. Adding to existing knowledge of political communication on Twitter \cite{Aragn2013CommunicationDI, Theocharis2020TheDO, Hemphill2013WhatsCD, Straus2013CommunicatingI1}, we sought to offer more holistic insights by incorporating the nuanced influences of not just Party and Chamber, but also the size of the Represented State - a facet relatively less examined \cite{Chamberlain2021ANA, Valle2021PoliticalIB}.

Our methodology entailed constructing a directed graph representing Twitter interactions among Congress members over four months, which were then analysed using a blend of descriptive statistics and regression models. Unlike preceding studies that have largely focused on party-centric engagements \cite{Aragn2013CommunicationDI, Chamberlain2021ANA, Valle2021PoliticalIB}, our principal finding has been the significant role played by the size of state representation.

While the conclusions drawn add value to our understanding of political dynamics on Twitter, there are certain limitations. The data spanned only four months, potentially overlooking the intricacies of interactions over the long-term. Further, the analysis failed to differentiate between the types of engagements (such as 'likes', 'retweets', or 'replies'), and ignored the context or content of these interactions. Also, the minimum tweet threshold set for inclusion in the database may have inadvertently excluded less frequent, yet active, Twitter users from the sample.

Our findings have implications for the broader understanding of political dynamics within Congress. The revelation that attributes beyond party lines, such as a state’s representation size, can influence Twitter interactions - thereby indirectly impacting information dissipation and nurturing of political ideas – advocates for a reconsideration of the current understanding of political discourses \cite{Hemphill2013WhatsCD}. It underscores the need for inclusivity of these subtleties when planning social media strategies for political communication, particularly in our current age characterized by digital transformation and polarized political landscapes.

It paves the way for several fascinating avenues for prospective research. Diving into a qualitative analysis of Twitter interactions, including aspects such as sentiment and content, could unearth valuable details about these political discourses. Exploring a diverse range of social media platforms apart from Twitter could offer a broader perspective on these interactions \cite{Borge-Holthoefer2014ContentAN}. Furthermore, integrating the insights drawn from our study with research focusing on the tone and tenor of online political conversations \cite{Rathje2021OutgroupAD} might throw light on how language constructs vary with factors such as the size of state representation, and thereby aid in the development of a comprehensive understanding of online political discourses.

In summary, this study has helped demystify some of the nuances in online political discourse among US Congress members. It has underscored the sensitivity of these discourses to factors beyond party-lines and chambers, such as State representation size, thus broadening the purview of political communication studies, and indicating promising paths for future investigations.

\section*{Methods}

\subsection*{Data Source}
The data used in this study consists of two files: "congress\_members.csv" and "congress\_edges.dat". The "congress\_members.csv" file provides information about the members of the 117th Congress, including their Twitter handles, Represented State, Party, and Chamber. The "congress\_edges.dat" file represents the interaction network between Congress members on Twitter, with each line containing two integers to indicate a directed edge from one Congress member to another.

\subsection*{Data Preprocessing}
The data preprocessing steps were performed using Python. First, the "congress\_edges.dat" file was loaded into a networkx graph object to represent the interaction network between Congress members. The "congress\_members.csv" file was also loaded into a pandas DataFrame to store the information about each Congress member.

In order to calculate the engagement count for each Congress member, a dictionary was created to map each node ID to its out-degree in the graph, representing the number of Twitter interactions. This engagement count was then added as a new column in the DataFrame.

To investigate the influence of State representation size on Twitter interactions, a new column was created in the DataFrame to indicate the number of representatives from each State. This was done by counting the occurrences of each State in the "State" column and assigning the count to each corresponding Congress member.

NaN values resulting from the preprocessing steps were dropped from the DataFrame to ensure data integrity.

\subsection*{Data Analysis}
The analysis of the data was performed using various statistical techniques in Python. 

First, a descriptive statistics table was created to summarize the mean and standard deviation of Twitter interactions of Congress members grouped by Party and Chamber.

Next, a statistical model was fitted to examine the effect of State representation size on Twitter interactions while controlling for Party and Chamber. The model used the engagement count as the dependent variable and included State representation size, Party, and Chamber as independent variables. The statistical model was fitted using the Ordinary Least Squares (OLS) method from the statsmodels library.

The results from the fitted model were used to create a table displaying the coefficients, p-values, and confidence intervals of the independent variables. Additionally, the model's R-squared value was calculated to assess the goodness of fit.

In summary, the data analysis involved descriptive statistics to understand the distribution of Twitter interactions among Congress members, and a regression analysis to examine the influence of State representation size on engagement levels while controlling for Party and Chamber.\subsection*{Code Availability}

Custom code used to perform the data preprocessing and analysis, as well as the raw code outputs, are provided in Supplementary Methods.


\clearpage
\appendix

\section{Data Description} \label{sec:data_description} Here is the data description, as provided by the user:

\begin{Verbatim}[tabsize=4]
* Rationale:
The dataset maps US Congress's Twitter interactions into a directed graph with
	social interactions (edges) among Congress members (nodes). Each member (node)
	is further characterized by three attributes: Represented State, Political
	Party, and Chamber, allowing analysis of the adjacency matrix structure, graph
	metrics and likelihood of interactions across these attributes.

* Data Collection and Network Construction:
Twitter data of members of the 117th US Congress, from both the House and the
	Senate, were harvested for a 4-month period, February 9 to June 9, 2022 (using
	the Twitter API). Members with fewer than 100 tweets were excluded from the
	network.

- `Nodes`. Nodes represent Congress members. Each node is designated an integer
	node ID (0, 1, 2, ...) which corresponds to a row in `congress_members.csv`,
	providing the member's Represented State, Political Party, and Chamber.

- `Edges`. A directed edge from node i to node j indicates that member i engaged
	with member j on Twitter at least once during the 4-month data-collection
	period. An engagement is defined as a tweet by member i that mentions member j's
	handle, or as retweets, quote tweets, or replies of i to a tweet by member j.


* Data analysis guidelines:
- Your analysis code should NOT create tables that include names of Congress
	members, or their Twitter handles.
- Your analysis code should NOT create tables that include names of States, or
	their two-letter abbreviations. The code may of course do statistical analysis
	of *properties* related to States, but should not single out specific states.


2 data files:
File #1: "congress_members.csv"
A csv file of members of the 117th Congress, including their Twitter handles,
	Represented State, Party, and Chamber.
Data source: `https://pressgallery.house.gov/member-data/members-official-
	twitter-handles`.
Rows are ordered according to the node ID, starting at 0.

Fields:

`Handle`: Twitter handle (without `@`)
`State`: Categorical; Two-letter state abbreviation; including also: "DC", "PR",
	"VI", "AS", "GU", "MP".
`Party`: Categorical; Party affiliation ("D", "R", or "I")
`Chamber`: Categorical; The member's chamber ("House", "Senate")


Here are the first few lines of the file:
```output
Handle,State,Party,Chamber
SenatorBaldwin,WI,D,Senate
SenJohnBarrasso,WY,R,Senate
SenatorBennet,CO,D,Senate

```

File #2: "congress_edges.dat"
This file provides the interaction network between members of the 115th US
	Congress on Twitter.
Download and adapted from: `https://snap.stanford.edu/data/congress-twitter`

Each line contains two integers (i, j), indicating a directed edge from node ID
	i to node ID j, compatible with nx.read_edgelist('congress_edges.dat',
	create_using=nx.DiGraph()). An i->j edge indicates that Congress member i had at
	least one tweet engaging with Congress member j during the 4-month collection
	period.




\end{Verbatim}

\section{Data Exploration}
\subsection{{Code}}
The Data Exploration was carried out using the following custom code:

\begin{minted}[linenos, breaklines]{python}

import pandas as pd
import numpy as np
import networkx as nx

# load data
members = pd.read_csv('congress_members.csv')
edges = nx.read_edgelist('congress_edges.dat', create_using=nx.DiGraph())

# open the output file
with open('data_exploration.txt', 'w') as f:

    ########## DATA SIZE #############
    f.write('# Data Size\n')
    f.write(f'Number of nodes (Congress members): {len(members)}\n')
    f.write(f'Number of edges (Interactions): {edges.number_of_edges()}\n')
    f.write('\n')

    ########## SUMMARY STATISTICS #############
    f.write('# Summary Statistics\n')
    f.write('## Members file:\n')
    f.write(members.describe(include='all').to_string())
    f.write('\n')
    f.write('## Edges file:\n')
    f.write(f'Number of interactions per node:\n')
    f.write(pd.Series([d for n, d in edges.degree()], name='Degree').describe().to_string())
    f.write('\n\n')

    ########## CATEGORICAL VARIABLES #############
    f.write('# Categorical Variables\n')
    f.write('## Members file:\n')
    for col in ['State', 'Party', 'Chamber']:
        f.write(f'{col} most common values:\n')
        f.write(members[col].value_counts().head().to_string())
        f.write('\n')
    f.write('\n')
    
    ########## MISSING VALUES #############
    f.write('# Missing Values\n')
    f.write('## Members file:\n')
    f.write(members.isna().sum().to_string())  
    f.write('\n\n')


\end{minted}

\subsection{Code Description}

The purpose of the code is to perform data exploration on the provided dataset of US Congress Twitter interactions.

First, the code loads the dataset by reading the "congress\_members.csv" file, which contains information about the Congress members, such as their Twitter handles, represented states, party affiliations, and chambers. It also reads the "congress\_edges.dat" file, which represents the interaction network between Congress members on Twitter.

The code then proceeds to perform several analysis steps and write the results into the "data\_exploration.txt" file.

1. Data Size:
   - The code calculates and writes the number of nodes (Congress members) and the number of edges (interactions) in the dataset.

2. Summary Statistics:
   - For the members file, the code computes and writes the descriptive statistics for all columns, including measures such as count, unique values, top value, and frequency.
   - For the edges file, the code calculates and writes the number of interactions per node, including the minimum, maximum, mean, standard deviation, and quartile values.

3. Categorical Variables:
   - For each of the categorical variables (State, Party, Chamber) in the members file, the code determines the most common values and writes them into the file.

4. Missing Values:
   - The code checks for missing values in the members file and writes the count of missing values for each column.

The output of the code is a text file, "data\_exploration.txt", which contains the results of the data exploration analysis. This file provides essential information and summary statistics about the dataset, allowing researchers to gain insights into the characteristics of the Congress Twitter interactions and make informed decisions for further analysis.

\subsection{Code Output}

\subsubsection*{data\_exploration.txt}

\begin{Verbatim}[tabsize=4]
# Data Size
Number of nodes (Congress members): 475
Number of edges (Interactions): 13289

# Summary Statistics
## Members file:
                Handle State Party Chamber
count              475   475   475     475
unique             475    54     3       2
top     SenatorBaldwin    CA     D   House
freq                 1    49   250     383
## Edges file:
Number of interactions per node:
count     475
mean    55.95
std     34.83
min         2
25%        33
50%        48
75%        69
max       284

# Categorical Variables
## Members file:
State most common values:
State
CA    49
TX    32
NY    27
FL    26
IL    17
Party most common values:
Party
D    250
R    223
I      2
Chamber most common values:
Chamber
House     383
Senate     92

# Missing Values
## Members file:
Handle     0
State      0
Party      0
Chamber    0


\end{Verbatim}

\section{Data Analysis}
\subsection{{Code}}
The Data Analysis was carried out using the following custom code:

\begin{minted}[linenos, breaklines]{python}

# IMPORT
import pandas as pd
import networkx as nx
import statsmodels.formula.api as smf
import pickle

# LOAD DATA
edges = nx.read_edgelist('congress_edges.dat', create_using=nx.DiGraph())
members = pd.read_csv('congress_members.csv')

# DATASET PREPARATIONS
# create a column indicating the engagement count (number of Twitter interactions) for each congress member
engagement_dict = {int(n): degree for n, degree in edges.out_degree()}
members["EngagementCount"] = members.index.map(engagement_dict).fillna(0)

# DESCRIPTIVE STATISTICS
# No descriptive statistics table is needed.

# PREPROCESSING 
# Create a new column indicating the size of representation for each congress member's state
state_counts = members['State'].value_counts()
members['StateRepresentation'] = members['State'].apply(lambda x: state_counts[x])

# Drop any NaN rows resulted from previous operations
members.dropna(inplace=True)

# ANALYSIS
## Table 1: "Mean and standard deviation of Twitter interactions of Congress members by Party and Chamber"
df1 = members.groupby(['Party', 'Chamber']).EngagementCount.agg(['mean', 'std'])
df1.to_pickle('table_1.pkl')

## Table 2: "ANOVA results for the effect of state representation size on Twitter interactions, controlling for party and chamber"
model = smf.ols('EngagementCount ~ StateRepresentation + C(Party) + C(Chamber)', data=members).fit()

# Extract the p-values, coefficients and confidence intervals from the fitted model
parameters = model.params
pvalues = model.pvalues
conf_ints = model.conf_int()
df2 = pd.concat([parameters, pvalues, conf_ints], axis=1)
df2.columns = ['coef', 'pvalue', 'CI_lower', 'CI_upper']
df2.to_pickle('table_2.pkl')

# SAVE ADDITIONAL RESULTS
additional_results = {
    'Total number of observations': len(members), 
    'Total number of interactions': members['EngagementCount'].sum(),
    'Model R-squared': model.rsquared
}
with open('additional_results.pkl', 'wb') as f:
    pickle.dump(additional_results, f)

\end{minted}

\subsection{Code Description}

The code performs an analysis on the dataset of US Congress Twitter interactions. The main steps of the analysis are as follows:

1. Load the data: The code reads two data files, "congress\_edges.dat" and "congress\_members.csv", which contain the Twitter interaction network between Congress members and information about each member, respectively. The network data is loaded into a directed graph object using the NetworkX library, and the member information is loaded into a pandas DataFrame.

2. Dataset preparations: The code calculates the engagement count (number of Twitter interactions) for each Congress member based on the network data and adds it as a column to the DataFrame. This allows for the analysis of the average and standard deviation of Twitter interactions by Party and Chamber in Table 1.

3. Preprocessing: The code creates a new column in the DataFrame indicating the size of representation for each Congress member's state. This is calculated by counting the number of members from each state and assigning it as the state representation size. Rows with missing values resulting from previous operations are dropped.

4. Analysis:
   a. Table 1: The code groups the DataFrame by Party and Chamber and calculates the mean and standard deviation of Twitter interactions for each group. The results are saved as a pickle file named "table\_1.pkl".
   
   b. Table 2: The code performs an analysis of variance (ANOVA) to examine the effect of state representation size on Twitter interactions, controlling for Party and Chamber. The model is fitted using the statsmodels library and the results, including coefficients, p-values, and confidence intervals, are saved as a pickle file named "table\_2.pkl".
   
5. Additional results: The code calculates additional results, including the total number of observations, total number of interactions, and the R-squared value of the model. These results are stored in a dictionary and saved as a pickle file named "additional\_results.pkl".

The code provides insight into the patterns of Twitter interactions among US Congress members, including the average and standard deviation of interactions by Party and Chamber, and the effect of state representation size on interactions while controlling for other factors. The additional results provide a summary of the dataset and model performance.

\subsection{Code Output}

\subsubsection*{table\_1.pkl}

\begin{Verbatim}[tabsize=4]
                    mean        std
Party Chamber
D     House    27.009852  21.179569
      Senate   32.085106  14.890659
I     Senate   25.000000  11.313708
R     House    28.355556  17.041724
      Senate   26.604651  11.582293
\end{Verbatim}

\subsubsection*{table\_2.pkl}

\begin{Verbatim}[tabsize=4]
                           coef     pvalue   CI_lower   CI_upper
Intercept             24.194696  3.121e-34  20.603907  27.785485
C(Party)[T.I]         -2.944532     0.8225 -28.717564  22.828500
C(Party)[T.R]          0.797416     0.6414  -2.564848   4.159681
C(Chamber)[T.Senate]   3.256861     0.1424  -1.098801   7.612523
StateRepresentation    0.164325    0.01183   0.036545   0.292105
\end{Verbatim}

\subsubsection*{additional\_results.pkl}

\begin{Verbatim}[tabsize=4]
{
    'Total number of observations': 475,
    'Total number of interactions': 13289,
    'Model R-squared': 0.01502             ,
}
\end{Verbatim}

\section{LaTeX Table Design}
\subsection{{Code}}
The LaTeX Table Design was carried out using the following custom code:

\begin{minted}[linenos, breaklines]{python}

# IMPORT
import pandas as pd
from my_utils import to_latex_with_note, format_p_value, is_str_in_df, split_mapping, AbbrToNameDef

# PREPARATION FOR ALL TABLES
shared_mapping: AbbrToNameDef = {
    'mean': ('Mean EC', 'Mean engagement count, number of Twitter interactions'),
    'std': ('Std EC', 'Standard deviation of engagement count'),
    'coef': ('Coef.', 'Coefficient from ANOVA'),
    'pvalue': (None, 'P-value'),
    'CI_lower': ('Lower 95% CI', 'Lower limit of the 95% confidence interval'),
    'CI_upper': ('Upper 95% CI', 'Upper limit of the 95% confidence interval'),
    'D': ('Democrat', None),
    'R': ('Republican', None),
    'I': ('Independent', None),
}

# TABLE 1:
df1 = pd.read_pickle('table_1.pkl')

# FORMAT VALUES
df1['mean'] = df1['mean'].apply(lambda x: round(x, 2))
df1['std'] = df1['std'].apply(lambda x: round(x, 2))

# RENAME ROWS AND COLUMNS
mapping1 = {k: v for k, v in shared_mapping.items() if is_str_in_df(df1, k)} 

abbrs_to_names1, legend1 = split_mapping(mapping1)
df1 = df1.rename(columns=abbrs_to_names1, index=abbrs_to_names1)

# Save as latex:
to_latex_with_note(
    df1, 'table_1.tex',
    caption="Mean and standard deviation of Twitter interactions of Congress members by Party and Chamber", 
    label='table:table_1',
    legend=legend1)

# TABLE 2:
df2 = pd.read_pickle('table_2.pkl')

# FORMAT VALUES 
df2['coef'] = df2['coef'].apply(lambda x: round(x, 3))
df2['pvalue'] = df2['pvalue'].apply(format_p_value)

# RENAME ROWS AND COLUMNS
mapping2 = {k: v for k, v in shared_mapping.items() if is_str_in_df(df2, k)}
mapping2.update({
    'Intercept': ('Intercept', 'ANOVA model intercept'),
    'C(Party)[T.I]': ('I Party', 'Independents compared with Democrats (reference group)'),
    'C(Party)[T.R]': ('R Party', 'Republicans compared with Democrats (reference group)'),
    'C(Chamber)[T.Senate]': ('Senate', 'Senate compared to House (reference group)'),
    'StateRepresentation': ('State Rep.', 'Number of Representatives from the state')
})

abbrs_to_names2, legend2 = split_mapping(mapping2)
df2 = df2.rename(columns=abbrs_to_names2, index=abbrs_to_names2)

# Save as latex:
to_latex_with_note(
    df2, 'table_2.tex',
    caption="ANOVA results for the effect of state representation size on Twitter interactions, controlling for party and chamber", 
    label='table:table_2',
    legend=legend2)

\end{minted}

\subsection{Provided Code}
The code above is using the following provided functions:

\begin{minted}[linenos, breaklines]{python}
def to_latex_with_note(df, filename: str, caption: str, label: str, note: str = None, legend: Dict[str, str] = None, **kwargs):
 """
 Converts a DataFrame to a LaTeX table with optional note and legend added below the table.

 Parameters:
 - df, filename, caption, label: as in `df.to_latex`.
 - note (optional): Additional note below the table.
 - legend (optional): Dictionary mapping abbreviations to full names.
 - **kwargs: Additional arguments for `df.to_latex`.

 Returns:
 - None: Outputs LaTeX file.
 """

def format_p_value(x):
 returns "{:.3g}".format(x) if x >= 1e-06 else "<1e-06"

def is_str_in_df(df: pd.DataFrame, s: str):
 return any(s in level for level in getattr(df.index, 'levels', [df.index]) + getattr(df.columns, 'levels', [df.columns]))

AbbrToNameDef = Dict[Any, Tuple[Optional[str], Optional[str]]]

def split_mapping(abbrs_to_names_and_definitions: AbbrToNameDef):
 abbrs_to_names = {abbr: name for abbr, (name, definition) in abbrs_to_names_and_definitions.items() if name is not None}
 names_to_definitions = {name or abbr: definition for abbr, (name, definition) in abbrs_to_names_and_definitions.items() if definition is not None}
 return abbrs_to_names, names_to_definitions

\end{minted}



\subsection{Code Output}

\subsubsection*{table\_1.tex}

\begin{Verbatim}[tabsize=4]
\begin{table}[h]
\caption{Mean and standard deviation of Twitter interactions of Congress members
	by Party and Chamber}
\label{table:table_1}
\begin{threeparttable}
\renewcommand{\TPTminimum}{\linewidth}
\makebox[\linewidth]{%
\begin{tabular}{llrr}
\toprule
 &  & Mean EC & Std EC \\
Party & Chamber &  &  \\
\midrule
\multirow[t]{2}{*}{\textbf{Democrat}} & \textbf{House} & 27 & 21.2 \\
\textbf{} & \textbf{Senate} & 32.1 & 14.9 \\
\cline{1-4}
\textbf{Independent} & \textbf{Senate} & 25 & 11.3 \\
\cline{1-4}
\multirow[t]{2}{*}{\textbf{Republican}} & \textbf{House} & 28.4 & 17 \\
\textbf{} & \textbf{Senate} & 26.6 & 11.6 \\
\cline{1-4}
\bottomrule
\end{tabular}}
\begin{tablenotes}
\footnotesize
\item \textbf{Mean EC}: Mean engagement count, number of Twitter interactions
\item \textbf{Std EC}: Standard deviation of engagement count
\end{tablenotes}
\end{threeparttable}
\end{table}

\end{Verbatim}

\subsubsection*{table\_2.tex}

\begin{Verbatim}[tabsize=4]
\begin{table}[h]
\caption{ANOVA results for the effect of state representation size on Twitter
	interactions, controlling for party and chamber}
\label{table:table_2}
\begin{threeparttable}
\renewcommand{\TPTminimum}{\linewidth}
\makebox[\linewidth]{%
\begin{tabular}{lrlrr}
\toprule
 & Coef. & pvalue & Lower 95\% CI & Upper 95\% CI \\
\midrule
\textbf{Intercept} & 24.2 & $<$1e-06 & 20.6 & 27.8 \\
\textbf{I Party} & -2.94 & 0.822 & -28.7 & 22.8 \\
\textbf{R Party} & 0.797 & 0.641 & -2.56 & 4.16 \\
\textbf{Senate} & 3.26 & 0.142 & -1.1 & 7.61 \\
\textbf{State Rep.} & 0.164 & 0.0118 & 0.0365 & 0.292 \\
\bottomrule
\end{tabular}}
\begin{tablenotes}
\footnotesize
\item \textbf{Coef.}: Coefficient from ANOVA
\item \textbf{pvalue}: P-value
\item \textbf{Lower 95\% CI}: Lower limit of the 95\% confidence interval
\item \textbf{Upper 95\% CI}: Upper limit of the 95\% confidence interval
\item \textbf{Intercept}: ANOVA model intercept
\item \textbf{I Party}: Independents compared with Democrats (reference group)
\item \textbf{R Party}: Republicans compared with Democrats (reference group)
\item \textbf{Senate}: Senate compared to House (reference group)
\item \textbf{State Rep.}: Number of Representatives from the state
\end{tablenotes}
\end{threeparttable}
\end{table}

\end{Verbatim}


\bibliographystyle{unsrt}
\bibliography{citations}

\end{document}
