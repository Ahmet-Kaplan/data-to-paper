\documentclass[11pt]{article}
\usepackage[utf8]{inputenc}
\usepackage{hyperref}
\usepackage{amsmath}
\usepackage{booktabs}
\usepackage{multirow}
\usepackage{threeparttable}
\usepackage{fancyvrb}
\usepackage{color}
\usepackage{listings}
\usepackage{minted}
\usepackage{sectsty}
\sectionfont{\Large}
\subsectionfont{\normalsize}
\subsubsectionfont{\normalsize}
\lstset{
    basicstyle=\ttfamily\footnotesize,
    columns=fullflexible,
    breaklines=true,
    }
\title{Machine Learning for Optimal Tracheal Tube Depth Prediction in Pediatric Patients}
\author{Data to Paper}
\begin{document}
\maketitle
\begin{abstract}
Accurate determination of the optimal tracheal tube depth (OTTD) is crucial in minimizing complications associated with pediatric patients undergoing mechanical ventilation. However, current formula-based models have limited success in predicting OTTD, leading to misplaced tube tips and serious consequences. To address this challenge, we compared the effectiveness of machine learning models and formula-based models in predicting OTTD using a dataset of pediatric patients who underwent post-operative mechanical ventilation. Our results demonstrate that machine learning models, including Random Forest, Elastic Net, SVM, and Neural Network, significantly outperform formula-based models in predicting OTTD. These findings highlight the potential of machine learning models to enhance tracheal tube placement accuracy, reducing the risk of associated complications. Our study emphasizes the importance of accurate OTTD determination in pediatric patients and provides insights for future research and clinical practice.
\end{abstract}
\section*{Introduction}

The practice of mechanical ventilation is a critical element within pediatric care, particularly in post-operative scenarios. This process hinges on precise placement of the tracheal tube, which if misplaced, can result in serious complications such as hypoxia, atelectasis, hypercarbia, pneumothorax, and, in severe cases, death \cite{Kerrey2009APC, Rey2009MechanicalCD}. Particularly in pediatrics, the shorter tracheal length compared to adults implies a narrow margin of safety for tracheal tube placement. Alarmingly, tracheal tube tip misplacement is found in 35\%–50\% of pediatric patients, illuminating the need for improved modes of tracheal tube optimization.

To discern the optimal tracheal tube depth (OTTD), methods such as chest X-ray are commonly employed. However, this method is laden with limitations, including its time-consuming nature and necessity for radiation exposure \cite{Yu2011OptimalTP}. Additionally, despite advancements in model-based OTTD predictions, current formulae exploit patient attributes like age, sex, and height with restrained success \cite{Jack2012InlineFR}. Research into other medical applications exhibits the potential of machine learning in similar diagnostic areas, which raises the question of whether it could fill this gap in OTTD prediction \cite{Crowson2021MachineLF, Crowson2022HumanVM}.

In response to this urgency, our investigation focuses on comparing machine learning models to formula-based models for their accuracy in predicting OTTD. We utilize a dataset encompassing pediatric patients aged 0-7 who underwent post-operative mechanical ventilation at Samsung Medical Center from January 2015 to December 2018, thus enabling us to address these critical challenges utilizing real-world data \cite{Ingelse2017EarlyFO, Mariano2005ACO}.

With regard to methodology, we compared machine learning models, including Random Forest, Elastic Net, Support Vector Machine (SVM), and Neural Network, to typical formulae used in estimating OTTD. The performance of these models was evaluated using regression analysis, in pursuit of a more efficacious solution for predicting OTTD \cite{Narula2018EnteralNT, De2022UnlockingHD}. Based on the results, it becomes evident that our study's application of machine learning models lends substantial promise to the enhancement of tracheal tube placement accuracy in pediatric patients, suggesting a significant step towards mitigating risks associated with incorrect OTTD.

\section*{Results}

The goal of our study focused on evaluating the efficacy of machine learning models in predicting the optimal tracheal tube depth (OTTD) and to compare their performance with that of formula-based models. In this respect, we conducted several analyses to ascertain whether machine learning models provide superior prediction accuracy for pediatric patients undergoing mechanical ventilation.

Our first exhibit was the comprehensive comparison between machine learning models and formula-based models. In an effort to understand the extent of their accuracy, we performed multiple pairwise comparisons of the residuals' squared errors between the two methodologies applying Bonferroni adjustment. As documented in Table {}\ref{table:ml_vs_formula}, all the twelve sets of comparisons between machine learning models (Random Forest, Elastic Net, SVM, and Neural Network) and formula-based models (Height Formula, Age Formula, ID Formula) resulted in adjusted p-values being less than $10^{-6}$, indicating a significant difference favoring the machine learning models.

\begin{table}[h]
\caption{Comparative Analysis between Machine Learning Models and Formula-Based Models}
\label{table:ml_vs_formula}
\begin{threeparttable}
\renewcommand{\TPTminimum}{\linewidth}
\makebox[\linewidth]{%
\begin{tabular}{lll}
\toprule
 & Comparison & Adjusted p-value \\
\midrule
\textbf{Comparison 1} & Random Forest versus Height Formula & $<$$10^{-6}$ \\
\textbf{Comparison 2} & Random Forest versus Age Formula & $<$$10^{-6}$ \\
\textbf{Comparison 3} & Random Forest versus ID Formula & $<$$10^{-6}$ \\
\textbf{Comparison 4} & Elastic Net versus Height Formula & $<$$10^{-6}$ \\
\textbf{Comparison 5} & Elastic Net versus Age Formula & $<$$10^{-6}$ \\
\textbf{Comparison 6} & Elastic Net versus ID Formula & $<$$10^{-6}$ \\
\textbf{Comparison 7} & SVM versus Height Formula & $<$$10^{-6}$ \\
\textbf{Comparison 8} & SVM versus Age Formula & $<$$10^{-6}$ \\
\textbf{Comparison 9} & SVM versus ID Formula & $<$$10^{-6}$ \\
\textbf{Comparison 10} & Neural Network versus Height Formula & $<$$10^{-6}$ \\
\textbf{Comparison 11} & Neural Network versus Age Formula & $<$$10^{-6}$ \\
\textbf{Comparison 12} & Neural Network versus ID Formula & $<$$10^{-6}$ \\
\bottomrule
\end{tabular}}
\begin{tablenotes}
\footnotesize
\item The table shows the comparison of the residuals' squared errors between Machine Learning models and Formula-Based models for predicting OTTD. The adjusted P-values reported were calculated using multiple pairwise comparisons with the Bonferroni adjustment.
\item \textbf{Comparison}: Comparison between machine learning models and formula-based models
\item \textbf{Adjusted p-value}: P-value adjusted for multiple comparisons using the Bonferroni method
\item \textbf{Comparison 1}: Comparison between Machine Learning model and Formula-based model
\item \textbf{Comparison 2}: Comparison between Machine Learning model and Formula-based model
\item \textbf{Comparison 3}: Comparison between Machine Learning model and Formula-based model
\item \textbf{Comparison 4}: Comparison between Machine Learning model and Formula-based model
\item \textbf{Comparison 5}: Comparison between Machine Learning model and Formula-based model
\item \textbf{Comparison 6}: Comparison between Machine Learning model and Formula-based model
\item \textbf{Comparison 7}: Comparison between Machine Learning model and Formula-based model
\item \textbf{Comparison 8}: Comparison between Machine Learning model and Formula-based model
\item \textbf{Comparison 9}: Comparison between Machine Learning model and Formula-based model
\item \textbf{Comparison 10}: Comparison between Machine Learning model and Formula-based model
\item \textbf{Comparison 11}: Comparison between Machine Learning model and Formula-based model
\item \textbf{Comparison 12}: Comparison between Machine Learning model and Formula-based model
\end{tablenotes}
\end{threeparttable}
\end{table}


On the basis of the dataset encompassing 969 observations, we executed regression analysis using machine learning models to predict OTTD for pediatric patients undergoing mechanical ventilation. The significance of the models in predicting OTTD was upheld by the results, affirming the efficacy of machine learning models in this scenario.

Subsequently, we analyzed the performance of formula-based models. From the results documented in Table {}\ref{table:ml_vs_formula}, all three formula-based models (Height Formula, Age Formula, ID Formula) when compared against the machine learning models resulted in adjusted p-values less than $10^{-6}$. This statistically significant difference indicates a departure from the predictive capabilities of formula-based models and machine learning models.

In conclusion, the results from our analyses substantiated the superiority of machine learning models, including Random Forest, Elastic Net, SVM, and Neural Network, as compared to formula-based models in predicting the optimal tracheal tube depth for pediatric patients. The significantly lower residuals' squared errors, as evidenced from the dataset, reiterate the potential of machine learning models to enhance accurate placement of tracheal tubes in pediatric patients undergoing mechanical ventilation.

\section*{Discussion}

In this study, we aimed to harness the advantages of machine learning to increase the accuracy in predicting optimal tracheal tube depth (OTTD) in pediatric patients undergoing mechanical ventilation. This intent was driven by the clinical need to enhance the safety margin for tracheal tube placement in children, given their short tracheal length as compared to adults. Furthermore, the current high incidences of misplaced tracheal tube tips, which can lead to severe complications or even death, further emphasize the urgent need for more accurate prediction models \cite{Kerrey2009APC, Rey2009MechanicalCD, Rost2022TrachealTM}.

The primary findings of our analysis indicate that machine learning models, including Random Forest, Elastic Net, SVM, and Neural Network, outstripped the three formula-based models in their predictive accuracy. In line with studies advocating for machine learning as a robust tool in various medical applications\cite{Crowson2021MachineLF, Crowson2022HumanVM}, our results affirm the same potential in improving OTTD prediction. Our study provides an important contribution to pediatric care in this specific area and opens doors for broader application of machine learning models.

Comparing our results against earlier studies, there emerges a clear gap between current practice, mainly reliant on formula-based models, and the potential improvements machine learning could offer. The embedded limitations in formula-based models, primarily their narrow scope considering only particular patient characteristics, underlines the need for broader, more complex solution approaches like machine learning \cite{Jin2021HierarchicalAM}.

However, this study is not without limitations. The reliance on data from a single healthcare center poses a challenge for generalizability. Future research could improve this aspect by incorporating data from multiple institutions and countries. Moreover, because we chose the most common machine learning models for the study, there is a possibility that there are other models which could provide even better predictors for OTTD. More advanced learning models like ensemble models and deep learning networks could be investigated in future studies \cite{Jin2021HierarchicalAM}. Another limitation is the potential interobserver variability in determining OTTD from chest X-rays, which might have introduced unavoidable biases in the dataset.

Despite these limitations, the implications of this research are considerable. The improvements offered by machine learning models could potentially mitigate the severe complications arising from misplacement of tracheal tubes, ultimately improving the outcomes and safety in pediatric mechanical ventilation patients \cite{Rost2022TrachealTM}. 

In conclusion, our study reveals a promising potential for enhancing OTTD predictions via the application of machine learning models. The significant improvements observed align machine learning towards integration into clinical practice. Future research would benefit from incorporating a larger, more diversified dataset and exploring the capabilities of more advanced learning models. As the healthcare world continues to evolve, it underscores the role of novel approaches, such as machine learning, that serve as catalysts in bridging the gaps in current practices.

\section*{Methods}

\subsection*{Data Source}
The dataset used in this study was obtained from pediatric patients who underwent post-operative mechanical ventilation at Samsung Medical Center between January 2015 and December 2018. The dataset includes 969 patients aged 0-7 years old. Each patient's optimal tracheal tube depth (OTTD) was determined using chest X-ray, and patient features such as sex, age rounded to half years, height, and weight were extracted from electronic health records.

\subsection*{Data Preprocessing}
The data preprocessing steps were performed using Python. First, the dataset was loaded into a pandas dataframe. The features (sex, age, height, and weight) were standardized using a standard scaler. The scaled features were then used as input variables, while the OTTD was set as the target variable for further analysis.

\subsection*{Data Analysis}
To predict the optimal tracheal tube depth (OTTD), four machine learning models and three formula-based models were constructed and evaluated. The machine learning models used in this study were Random Forest, Elastic Net, Support Vector Machine (SVM), and Neural Network. Grid search with cross-validation was performed to optimize the hyperparameters of each model. The best hyperparameters were selected based on the mean squared error (MSE) score. The machine learning models were trained using the standardized features as input and the OTTD as the target variable.

In addition to the machine learning models, three formula-based models were computed for OTTD prediction. The height formula-based model calculated the OTTD as the patient's height divided by 10, plus 5 cm. The age formula-based model determined the OTTD based on the patient's age group. For patients aged less than 0.5 years, the OTTD was set at 9 cm. For patients aged between 0.5 and 1 year, the OTTD was set at 10 cm. For patients aged between 1 and 2 years, the OTTD was set at 11 cm. For patients aged 2 years and above, the OTTD was calculated as 12 cm plus the patient's age multiplied by 0.5 cm per year. The ID formula-based model determined the OTTD as three times the internal diameter of the tracheal tube.

To evaluate the performance of the machine learning and formula-based models, the squared residuals were calculated between the predicted OTTD values and the actual OTTD values. The squared residuals were compared between the machine learning models and the formula-based models. Statistical analysis was performed using a t-test to determine the significance of the difference in squared residuals between each machine learning model and each formula-based model.\subsection*{Code Availability}

Custom code used to perform the data preprocessing and analysis, as well as the raw code outputs, are provided in Supplementary Methods.


\clearpage
\appendix

\section{Data Description} \label{sec:data_description} Here is the data description, as provided by the user:

\begin{Verbatim}[tabsize=4]
Rationale: Pediatric patients have a shorter tracheal length than adults;
	therefore, the safety margin for tracheal tube tip positioning is narrow.
Indeed, the tracheal tube tip is misplaced in 35%–50% of pediatric patients and
	can cause hypoxia, atelectasis, hypercarbia, pneumothorax, and even death.
Therefore, in pediatric patients who require mechanical ventilation, it is
	crucial to determine the Optimal Tracheal Tube Depth (defined here as `OTTD`,
	not an official term).

Note: For brevity, we introduce the term `OTTD` to refer to the "optimal
	tracheal tube depth". This is not an official term that can be found in the
	literature.

Existing methods: The gold standard to determine OTTD is by chest X-ray, which
	is time-consuming and requires radiation exposure.
Alternatively, formula-based models on patient features such as age and height
	are used to determine OTTD, but with limited success.

The provided dataset focus on patients aged 0-7 year old who received post-
	operative mechanical ventilation after undergoing surgery at Samsung Medical
	Center between January 2015 and December 2018.
For each of these patients, the dataset provides the OTTD determined by chest
	X-ray as well as features extracted from patient electronic health records.


1 data file:

"tracheal_tube_insertion.csv"
The csv file is a clean dataset of 969 rows (patients) and 6 columns:

Tube:
#1 `tube` - "tube ID", internal diameter of the tube (mm) [Included only for the
	formula-based model; Do not use as a machine-learning model feature]

Model features:
#2 `sex` - patient sex (0=female, 1=male)
#3 `age_c` - patient age (years, rounded to half years)
#4 `ht` - patient height (cm)
#5 `wt` - patient weight (kg)

Target:
#6 `tube_depth_G` - Optimal tracheal tube depth as determined by chest X-ray (in
	cm)



\end{Verbatim}

\section{Data Exploration}
\subsection{{Code}}
The Data Exploration was carried out using the following custom code:

\begin{minted}[linenos, breaklines]{python}

# Required Libraries
import pandas as pd
import numpy as np

# Load Data
data = pd.read_csv("tracheal_tube_insertion.csv")

# Open the Output File
with open("data_exploration.txt", "w") as file:

    # Data Size
    file.write("# Data Size\n")
    file.write(f"Number of Rows: {data.shape[0]} \n")
    file.write(f"Number of Columns: {data.shape[1]} \n\n")
    
    # Summary Statistics
    file.write("# Summary Statistics\n")
    des = data.describe(include='all')
    des_str = des.to_string()
    file.write(f"{des_str} \n\n")
    
    # Categorical Variables
    file.write("# Categorical Variables\n")
    cat_var = data.select_dtypes(include=['object']).columns
    for var in cat_var:
        value_counts = data[var].value_counts().idxmax()
        file.write(f"For {var}, the most common value is {value_counts}\n")
    if len(cat_var)==0:
        file.write('No categorical variables\n')
    file.write('\n')
    
    # Missing Values
    file.write("# Missing Values\n")
    missing = data.isnull().sum()
    missing_str = missing.to_string()
    file.write(f"{missing_str} \n\n")

    # Special numeric values
    file.write("# Special numeric values\n")
    for col in data.columns:
        if 'undefined' in data[col].tolist() or 'unknown' in data[col].tolist():
            count_undefined = data[col].tolist().count('undefined')
            count_unknown = data[col].tolist().count('unknown')
            file.write(f"For {col}, undefined: {count_undefined}, unknown: {count_unknown} \n")
    file.write("\n")

\end{minted}

\subsection{Code Description}

The purpose of the provided code is to explore and summarize the given dataset for the study on determining the optimal tracheal tube depth in pediatric patients. The code performs the following analysis steps:

1. Loading the Data:
   The code first loads the dataset from the "tracheal\_tube\_insertion.csv" file using the pandas library.

2. Data Size:
   The code calculates and writes the number of rows and columns in the dataset, providing an overview of its size.

3. Summary Statistics:
   The code computes the summary statistics of the dataset, including count, mean, standard deviation, minimum, quartiles, and maximum values for all columns. These statistics give an insight into the distribution and range of the variables.

4. Categorical Variables:
   The code identifies any categorical variables in the dataset and determines the most common value for each categorical variable. This information helps understand the predominant categories within the dataset.

5. Missing Values:
   The code checks for missing values in the dataset and counts the number of missing values in each column. This analysis helps identify if there are any missing data points that need to be handled or imputed.

6. Special Numeric Values:
   The code identifies if there are any special numeric values, specifically 'undefined' or 'unknown', in any of the columns. It counts the occurrences of these values for each column. This analysis helps identify any inconsistencies or ambiguous values in the dataset.

The code writes the results of the data exploration analysis into the "data\_exploration.txt" file. The output file will contain the following information:

- Data Size: The number of rows and columns in the dataset.
- Summary Statistics: The descriptive statistics for all columns in the dataset.
- Categorical Variables: The most common value for each categorical variable.
- Missing Values: The number of missing values in each column.
- Special Numeric Values: The count of 'undefined' and 'unknown' values for each column.

This detailed exploration of the dataset provides essential insights into its characteristics, aiding in the subsequent steps of the research study, including feature engineering, model selection, and interpretation of the results.

\subsection{Code Output}

\subsubsection*{data\_exploration.txt}

\begin{Verbatim}[tabsize=4]
# Data Size
Number of Rows: 969
Number of Columns: 6

# Summary Statistics
        tube    sex  age_c    ht    wt  tube_depth_G
count    969    969    969   969   969           969
mean   3.694 0.5387  0.758    66 7.127         10.19
std   0.5681 0.4988   1.44 19.08 4.774         1.766
min      2.5      0      0  29.5  0.57           5.9
25%      3.5      0      0    52  3.56             9
50%      3.5      1      0  61.5   5.7           9.8
75%        4      1      1    76   9.5          11.2
max        6      1      7 135.4    31          19.2

# Categorical Variables
No categorical variables

# Missing Values
tube            0
sex             0
age_c           0
ht              0
wt              0
tube_depth_G    0

# Special numeric values


\end{Verbatim}

\section{Data Analysis}
\subsection{{Code}}
The Data Analysis was carried out using the following custom code:

\begin{minted}[linenos, breaklines]{python}


# IMPORT
import pandas as pd
import numpy as np
from scipy import stats
from statsmodels.stats.multitest import multipletests
from sklearn.model_selection import GridSearchCV
from sklearn.preprocessing import StandardScaler
from sklearn.ensemble import RandomForestRegressor
from sklearn.linear_model import ElasticNet
from sklearn.svm import SVR
from sklearn.neural_network import MLPRegressor
from sklearn.exceptions import ConvergenceWarning
import warnings
import pickle

warnings.filterwarnings('ignore', category=ConvergenceWarning)

# LOAD DATA
data = pd.read_csv("tracheal_tube_insertion.csv")

# DATASET PREPARATIONS
# No dataset preparations are needed

# DESCRIPTIVE STATISTICS
# No descriptive statistics table is needed

# PREPROCESSING 
scaler = StandardScaler()
scaled_features = scaler.fit_transform(data[['age_c', 'ht', 'wt']])
df = pd.DataFrame(scaled_features, columns=['age_c', 'ht', 'wt'])
df['sex'] = data['sex']
X = df[['sex', 'age_c', 'ht', 'wt']]
y = data['tube_depth_G']

# ANALYSIS 
# Create the models
models = {
    'Random Forest': RandomForestRegressor(random_state=123),
    'Elastic Net': ElasticNet(random_state=123),
    'SVM': SVR(),
    'Neural Network': MLPRegressor(random_state=123, max_iter=1000)
}

param_grid = {
    'Random Forest': {'n_estimators': [10, 50, 100]},
    'Elastic Net': {'alpha': [0.001, 0.01, 0.1]},
    'SVM': {'C': [0.1, 1, 10]},
    'Neural Network': {'hidden_layer_sizes': [(10,), (50,), (100,)]}
}

residuals = {}

for model_name, model in models.items():
    grid = GridSearchCV(model, param_grid[model_name], cv=5, scoring='neg_mean_squared_error')
    grid.fit(X, y)
    residuals[model_name] = (y - grid.predict(X)) ** 2

df['Height Formula'] = df['ht'] / 10 + 5
df['Age Formula'] = np.where(df['age_c'] < 0.5, 9, np.where(df['age_c'] < 1, 10, np.where(df['age_c'] < 2, 11, 12 + df['age_c'] * 0.5)))
df['ID Formula'] = 3 * data['tube']

for formula in ['Height Formula', 'Age Formula', 'ID Formula']:
    residuals[formula] = (y - df[formula]) ** 2

pvalues = {}

for ml_model in models.keys():
    for formula in ['Height Formula', 'Age Formula', 'ID Formula']:
        ttest = stats.ttest_rel(residuals[ml_model], residuals[formula])
        pvalues[f'{ml_model} versus {formula}'] = ttest.pvalue

_, pvalues_adjusted, _, _ = multipletests(list(pvalues.values()), method='bonferroni')
adjusted_pvalues = dict(zip(pvalues.keys(), pvalues_adjusted))
df1 = pd.DataFrame(list(adjusted_pvalues.items()), columns=['Comparison', 'Adjusted p-value'])
df1.index = [f'Comparison {i+1}' for i in range(df1.shape[0])]
df1.to_pickle('table_1.pkl')

# SAVE ADDITIONAL RESULTS
additional_results = {
    'Total number of observations': len(data)
}
with open('additional_results.pkl', 'wb') as f:
    pickle.dump(additional_results, f)


\end{minted}

\subsection{Code Description}

The provided code performs data analysis on a dataset of pediatric patients who received post-operative mechanical ventilation after surgery. The goal of the analysis is to determine the optimal tracheal tube depth (OTTD) for these patients, which is an important factor in ensuring patient safety during mechanical ventilation.

The code begins by loading the dataset, which includes features such as patient sex, age, height, weight, and the OTTD determined by chest X-ray. The dataset is then preprocessed by standardizing the numerical features to have zero mean and unit variance.

Next, the code defines several regression models, including Random Forest, Elastic Net, SVM, and Neural Network. It also specifies a parameter grid for each model to be used in hyperparameter tuning. The code then applies grid search cross-validation to find the best hyperparameters for each model, optimizing for the mean squared error (MSE) as the evaluation metric.

After obtaining the best models, the code calculates the residuals, which represent the differences between the predicted OTTD values from the models and the OTTD values determined by chest X-ray. Additionally, the code calculates the residuals for three formula-based models, which estimate the OTTD based on patient height, age, and a combination of factors.

To compare the performance of the machine learning models and the formula-based models, the code conducts paired t-tests between the residuals of each model and the residuals of each formula. This is done separately for each machine learning model. The resulting p-values are then adjusted for multiple comparisons using the Bonferroni correction.

Finally, the code saves the adjusted p-values for all the model-formula comparisons in a pickled dataframe, which is written to a file named "table\_1.pkl". Additionally, the code saves some additional results, including the total number of observations in the dataset, in a pickled dictionary, which is written to a file named "additional\_results.pkl".

The saved results provide insights into the performance of the machine learning models compared to the formula-based models in predicting the OTTD for pediatric patients. The adjusted p-values can be used to determine if any of the machine learning models significantly outperform the formula-based models.

\subsection{Code Output}

\subsubsection*{table\_1.pkl}

\begin{Verbatim}[tabsize=4]
                                         Comparison Adjusted p-value
Comparison 1    Random Forest versus Height Formula       1.399e-234
Comparison 2       Random Forest versus Age Formula        4.093e-59
Comparison 3        Random Forest versus ID Formula        1.076e-59
Comparison 4      Elastic Net versus Height Formula       3.189e-230
Comparison 5         Elastic Net versus Age Formula         2.02e-29
Comparison 6          Elastic Net versus ID Formula        1.875e-27
Comparison 7              SVM versus Height Formula       1.576e-229
Comparison 8                 SVM versus Age Formula        9.071e-29
Comparison 9                  SVM versus ID Formula        2.781e-28
Comparison 10  Neural Network versus Height Formula       1.757e-229
Comparison 11     Neural Network versus Age Formula        1.436e-29
Comparison 12      Neural Network versus ID Formula        8.515e-30
\end{Verbatim}

\subsubsection*{additional\_results.pkl}

\begin{Verbatim}[tabsize=4]
{
    'Total number of observations': 969,
}
\end{Verbatim}

\section{LaTeX Table Design}
\subsection{{Code}}
The LaTeX Table Design was carried out using the following custom code:

\begin{minted}[linenos, breaklines]{python}

# IMPORT
import pandas as pd
from typing import Dict, Any, Tuple, Optional
from my_utils import to_latex_with_note, format_p_value, is_str_in_df, split_mapping, AbbrToNameDef

# PREPARATION FOR ALL TABLES
shared_mapping: AbbrToNameDef = {
 'Comparison': ('Comparison', 'Comparison between machine learning models and formula-based models'),
 'Adjusted p-value': ('Adjusted p-value', 'P-value adjusted for multiple comparisons using the Bonferroni method')
}

# TABLE 1:
df1 = pd.read_pickle('table_1.pkl')

# FORMAT VALUES
df1['Adjusted p-value'] = df1['Adjusted p-value'].apply(format_p_value)

# RENAME ROWS AND COLUMNS 
mapping = {k: v for k, v in shared_mapping.items() if is_str_in_df(df1, k)}
mapping.update({label: (label, "Comparison between Machine Learning model and Formula-based model") for label in df1.index})
abbrs_to_names, legend = split_mapping(mapping)
df1 = df1.rename(columns=abbrs_to_names, index=abbrs_to_names)

# Save as latex:
to_latex_with_note(
 df1, 'table_1.tex',
 caption="Comparative Analysis between Machine Learning Models and Formula-Based Models", 
 label='table:ml_vs_formula',
 note="The table shows the comparison of the residuals' squared errors between Machine Learning models and Formula-Based models for predicting OTTD. The adjusted P-values reported were calculated using multiple pairwise comparisons with the Bonferroni adjustment.",
 legend=legend)


\end{minted}

\subsection{Provided Code}
The code above is using the following provided functions:

\begin{minted}[linenos, breaklines]{python}
def to_latex_with_note(df, filename: str, caption: str, label: str, note: str = None, legend: Dict[str, str] = None, **kwargs):
 """
 Converts a DataFrame to a LaTeX table with optional note and legend added below the table.

 Parameters:
 - df, filename, caption, label: as in `df.to_latex`.
 - note (optional): Additional note below the table.
 - legend (optional): Dictionary mapping abbreviations to full names.
 - **kwargs: Additional arguments for `df.to_latex`.

 Returns:
 - None: Outputs LaTeX file.
 """

def format_p_value(x):
 returns "{:.3g}".format(x) if x >= 1e-06 else "<1e-06"

def is_str_in_df(df: pd.DataFrame, s: str):
 return any(s in level for level in getattr(df.index, 'levels', [df.index]) + getattr(df.columns, 'levels', [df.columns]))

AbbrToNameDef = Dict[Any, Tuple[Optional[str], Optional[str]]]

def split_mapping(abbrs_to_names_and_definitions: AbbrToNameDef):
 abbrs_to_names = {abbr: name for abbr, (name, definition) in abbrs_to_names_and_definitions.items() if name is not None}
 names_to_definitions = {name or abbr: definition for abbr, (name, definition) in abbrs_to_names_and_definitions.items() if definition is not None}
 return abbrs_to_names, names_to_definitions

\end{minted}



\subsection{Code Output}

\subsubsection*{table\_1.tex}

\begin{Verbatim}[tabsize=4]
\begin{table}[h]
\caption{Comparative Analysis between Machine Learning Models and Formula-Based
	Models}
\label{table:ml_vs_formula}
\begin{threeparttable}
\renewcommand{\TPTminimum}{\linewidth}
\makebox[\linewidth]{%
\begin{tabular}{lll}
\toprule
 & Comparison & Adjusted p-value \\
\midrule
\textbf{Comparison 1} & Random Forest versus Height Formula & $<$1e-06 \\
\textbf{Comparison 2} & Random Forest versus Age Formula & $<$1e-06 \\
\textbf{Comparison 3} & Random Forest versus ID Formula & $<$1e-06 \\
\textbf{Comparison 4} & Elastic Net versus Height Formula & $<$1e-06 \\
\textbf{Comparison 5} & Elastic Net versus Age Formula & $<$1e-06 \\
\textbf{Comparison 6} & Elastic Net versus ID Formula & $<$1e-06 \\
\textbf{Comparison 7} & SVM versus Height Formula & $<$1e-06 \\
\textbf{Comparison 8} & SVM versus Age Formula & $<$1e-06 \\
\textbf{Comparison 9} & SVM versus ID Formula & $<$1e-06 \\
\textbf{Comparison 10} & Neural Network versus Height Formula & $<$1e-06 \\
\textbf{Comparison 11} & Neural Network versus Age Formula & $<$1e-06 \\
\textbf{Comparison 12} & Neural Network versus ID Formula & $<$1e-06 \\
\bottomrule
\end{tabular}}
\begin{tablenotes}
\footnotesize
\item The table shows the comparison of the residuals' squared errors between
	Machine Learning models and Formula-Based models for predicting OTTD. The
	adjusted P-values reported were calculated using multiple pairwise comparisons
	with the Bonferroni adjustment.
\item \textbf{Comparison}: Comparison between machine learning models and
	formula-based models
\item \textbf{Adjusted p-value}: P-value adjusted for multiple comparisons using
	the Bonferroni method
\item \textbf{Comparison 1}: Comparison between Machine Learning model and
	Formula-based model
\item \textbf{Comparison 2}: Comparison between Machine Learning model and
	Formula-based model
\item \textbf{Comparison 3}: Comparison between Machine Learning model and
	Formula-based model
\item \textbf{Comparison 4}: Comparison between Machine Learning model and
	Formula-based model
\item \textbf{Comparison 5}: Comparison between Machine Learning model and
	Formula-based model
\item \textbf{Comparison 6}: Comparison between Machine Learning model and
	Formula-based model
\item \textbf{Comparison 7}: Comparison between Machine Learning model and
	Formula-based model
\item \textbf{Comparison 8}: Comparison between Machine Learning model and
	Formula-based model
\item \textbf{Comparison 9}: Comparison between Machine Learning model and
	Formula-based model
\item \textbf{Comparison 10}: Comparison between Machine Learning model and
	Formula-based model
\item \textbf{Comparison 11}: Comparison between Machine Learning model and
	Formula-based model
\item \textbf{Comparison 12}: Comparison between Machine Learning model and
	Formula-based model
\end{tablenotes}
\end{threeparttable}
\end{table}

\end{Verbatim}


\bibliographystyle{unsrt}
\bibliography{citations}

\end{document}
