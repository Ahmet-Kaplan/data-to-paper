\documentclass[11pt]{article}
\usepackage[utf8]{inputenc}
\usepackage{hyperref}
\usepackage{amsmath}
\usepackage{booktabs}
\usepackage{multirow}
\usepackage{threeparttable}
\usepackage{fancyvrb}
\usepackage{color}
\usepackage{listings}
\usepackage{minted}
\usepackage{sectsty}
\sectionfont{\Large}
\subsectionfont{\normalsize}
\subsubsectionfont{\normalsize}
\lstset{
    basicstyle=\ttfamily\footnotesize,
    columns=fullflexible,
    breaklines=true,
    }
\author{Data to Paper}
\begin{document}
% Define the save box within the document block
\newsavebox{\mytablebox} % Create a box to store the table

% Save only the tabular part of table in the \mytablebox without typesetting it:
\begin{lrbox}{\mytablebox}
 \begin{tabular}{lr}
\toprule
{} &  Mean Square Error \\
\textbf{Model                    } &                    \\
\midrule
\textbf{Random Forest            } &               1.54 \\
\textbf{Elastic Net              } &               1.28 \\
\textbf{Support Vector           } &               1.28 \\
\textbf{Artificial Neural Network} &               1.52 \\
\bottomrule
\end{tabular}%
\end{lrbox}

% Typeset the entire table:
\begin{table}[h]
\caption{Comparison of machine learning models for predicting optimal tracheal tube depth}
\label{table:model_comparisons}
\begin{threeparttable}
\renewcommand{\TPTminimum}{\linewidth}
\makebox[\linewidth]{%
\begin{tabular}{lr}
\toprule
{} &  Mean Square Error \\
\textbf{Model                    } &                    \\
\midrule
\textbf{Random Forest            } &               1.54 \\
\textbf{Elastic Net              } &               1.28 \\
\textbf{Support Vector           } &               1.28 \\
\textbf{Artificial Neural Network} &               1.52 \\
\bottomrule
\end{tabular}}
\begin{tablenotes}
\footnotesize
\item The models are compared based on their Mean Square Error (MSE) performance metric.
\item \textbf{Mean Square Error}: Mean square error of the model
\item \textbf{Random Forest}: Random forest regression based model
\item \textbf{Elastic Net}: Elastic net linear regression based model
\item \textbf{Support Vector}: Support vector regression based model
\item \textbf{Artificial Neural Network}: Artificial neural network based model
\end{tablenotes}
\end{threeparttable}
\end{table}


% Print the width of the tabular part of the table and the width of the page margin to the log file
\typeout{Table width: \the\wd\mytablebox}
\typeout{Page margin width: \the\textwidth}
\end{document}
