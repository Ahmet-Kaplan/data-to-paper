\documentclass[11pt]{article}
\usepackage[utf8]{inputenc}
\usepackage{hyperref}
\usepackage{amsmath}
\usepackage{booktabs}
\usepackage{multirow}
\usepackage{threeparttable}
\usepackage{fancyvrb}
\usepackage{color}
\usepackage{listings}
\usepackage{minted}
\usepackage{sectsty}
\sectionfont{\Large}
\subsectionfont{\normalsize}
\subsubsectionfont{\normalsize}
\lstset{
    basicstyle=\ttfamily\footnotesize,
    columns=fullflexible,
    breaklines=true,
    }
\author{Data to Paper}
\begin{document}
\section*{Code explanation}
The provided code performs an initial data exploration on the given dataset. Here is a high-level overview of the steps taken:

1. The necessary libraries are imported: pandas is used for data manipulation and numpy is used for numerical computations.

2. The dataset is loaded into a pandas dataframe named 'data' using the 'read\_csv' function.

3. The code creates a file called "data\_exploration.txt" for writing the output of the data exploration analysis.

4. The scale of the dataset is measured and written to the file. This includes the number of rows and columns in the dataset.

5. Summary statistics of the dataset are calculated and written to the file using the 'describe' function. This provides information such as count, mean, standard deviation, minimum, quartiles, and maximum values for each numerical feature.

6. Categorical variables in the dataset, specifically the 'sex' variable, are examined. The code calculates the frequency of each category within the 'sex' variable using the 'value\_counts' function. The results are written to the file.

7. Missing values in the dataset are checked using the 'isnull().sum()' function. This counts the number of missing values in each feature and writes the results to the file.

8. Finally, the file is closed, completing the data exploration analysis.

The "data\_exploration.txt" file contains the following information:

- The number of rows and columns in the dataset.
- Summary statistics for each numerical feature in the dataset.
- The frequency counts for the categorical variable 'sex'.
- The number of missing values in each feature of the dataset.

This code provides an initial understanding of the dataset, allowing researchers to identify any potential issues, such as missing values or the need for data preprocessing, before further analysis.

\end{document}
